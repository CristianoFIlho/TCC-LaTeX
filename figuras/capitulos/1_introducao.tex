\chapter{Introdução}\label{chp:intro}
A educação inclusiva é uma meta fundamental na busca por igualdade de oportunidades para os estudantes com necessidades especiais, levando em consideração suas habilidades e necessidades individuais. No contexto da inclusão educacional, crianças com Síndrome de Down têm sido objeto de estudo e atenção especial, devido aos desafios enfrentados em seu processo de aprendizagem.

A Síndrome de Down é uma condição genética causada pela presença de um cromossomo extra no par 21, conhecido como trissomia do 21. Ela afeta aproximadamente 1 em cada 700 nascimentos no Brasil e pode variar em gravidade, mas geralmente está associada a características físicas distintas, diferenças cognitivas e comunicação verbal afetada \cite{mariahelenavarellabruna2011}. 

A trissomia do 21 resulta de uma cópia extra do cromossomo 21 no embrião. Isso leva ao desenvolvimento das características físicas e cognitivas observadas em indivíduos com Síndrome de Down. Essas características incluem olhos amendoados, atrasos no desenvolvimento, hipotonia muscular e, frequentemente, uma variedade de condições médicas associadas, como problemas cardíacos, problemas de visão e audição, juntamente com desafios no desenvolvimento da linguagem e da fala \cite{mariahelenavarellabruna2011}. Em geral, as crianças com Síndrome de Down podem ser menores em tamanho e apresentar um desenvolvimento físico, mental e intelectual mais lento em comparação com outras crianças da mesma idade.

Em cada célula do indivíduo existe um total de 46 cromossomos, divididos em 23 pares. A Síndrome de Down é gerada pela presença de uma terceira cópia do cromossomo 21 em todas as células do organismo (trissomia). Isso ocorre no momento da concepção de uma criança, levando ao total de 47 cromossomos em suas células, em vez de 46, como na maior parte da população \cite{mariahelenavarellabruna2011}. Cromossomos são as estruturas biológicas que contêm as informações genéticas.

É fundamental que bebês e crianças com Síndrome de Down recebam acompanhamento médico precoce para identificar possíveis problemas cardíacos, gastrointestinais, endócrinos, auditivos e visuais. O tratamento iniciado de forma precoce pode prevenir complicações futuras \cite{mariahelenavarellabruna2011}.

Além disso, crianças com Síndrome de Down precisam ser estimuladas desde o nascimento para que possam superar as limitações impostas por essa alteração genética. Como têm necessidades específicas de saúde e aprendizagem, requerem assistência profissional multidisciplinar e atenção contínua dos pais. O objetivo principal é habilitá-las para o convívio e a participação social \cite{mariahelenavarellabruna2011}. 

Devido às diferenças cognitivas e de aprendizado associadas à Síndrome de Down, é essencial adotar abordagens educacionais adaptadas que possibilitem não apenas a aquisição de conhecimento acadêmico, mas também o desenvolvimento de habilidades essenciais para a vida cotidiana. Nesse contexto, a gamificação surge como uma abordagem promissora para o ensino de matemática básica devido ao seu potencial atrativo e estimulante.

Surge, então, a hipótese que buscamos verificar: por que o desenvolvimento do aplicativo gamificado, é relevante e necessário para crianças com Síndrome de Down?

\section{Aplicabilidade e Motivação}\label{chp:aplic}
Este trabalho de conclusão de curso tem como objetivo desenvolver um aplicativo gamificado dedicado ao ensino de matemática básica para crianças com Síndrome de Down, com idades entre sete e quinze anos. A principal inovação deste projeto é a incorporação ...... A aplicação dessa funcionalidade não apenas fortalece as habilidades matemáticas, mas também desempenha um papel fundamental no desenvolvimento ..... aspectos cruciais para a melhoria da ..... 

Existem duas motivações para este projeto. Em primeiro lugar, visa-se proporcionar uma abordagem educacional inclusiva e estimulante que contribua para o desenvolvimento acadêmico e social dessas crianças. Em segundo lugar, busca-se promover uma transformação significativa em suas vidas cotidianas, aumentando sua independência e confiança. Acreditamos que o uso de tecnologia e gamificação pode ser uma ferramenta poderosa para alcançar esses objetivos. 

Nesse contexto decidimos continuar um trabalho existente relacionado ao auxílio do ensino da matemática básica para crianças com Síndrome de Down. A motivação principal foi a inserção de novas funcionalidades ao aplicativo existente, que visam melhorar a experiência do usuário e expandir o escopo do aprendizado. 

Com isso, espera-se aprimorar a experiência do usuário e fornecer uma tecnologia assistiva eficiente para apoiar o aprendizado matemático desse público-alvo. As funcionalidades adicionadas visam proporcionar um processo de aprendizagem adaptativo, lúdico e inclusivo.

\section{Revisão da Literatura}

Esta introdução é complementada por uma revisão da literatura que aborda as melhores práticas em educação inclusiva, o papel da gamificação no ensino e os desafios específicos enfrentados por crianças com Síndrome de Down. Além disso, é importante destacar que este trabalho se baseia e expande uma pesquisa realizada na Universidade Católica do Salvador (UCSal) em 2023. Intitulado "Desenvolvimento de um aplicativo gamificado para o ensino de matemática básica para crianças com Síndrome de Down: uma abordagem utilizando elementos de gamificação como estratégia educacional" \cite{2023gamificacao}, estabeleceu as bases para a abordagem gamificada neste projeto. 

Ao longo deste trabalho, examinaremos como a combinação desses elementos pode resultar em uma abordagem eficaz para o ensino de matemática básica e o desenvolvimento de habilidades de ..... em crianças com Síndrome de Down, expandindo e aprimorando as ideias e objetivos do trabalho \cite{2023gamificacao} anteriormente citado. 

O Objetivo Geral do trabalho \cite{2023gamificacao} era desenvolver um aplicativo gamificado voltado para o ensino de matemática básica, com foco no desenvolvimento do ensino de matemática para crianças com Síndrome de Down por meio da gamificação. Este objetivo geral destaca a importância de proporcionar uma abordagem eficaz para o aprendizado da matemática nesse público. Os Objetivos Específicos incluíam melhorar o raciocínio lógico, apoiar o ensino da matemática, validar a eficácia dos jogos educativos e aprimorar o ensino da matemática para crianças com Síndrome de Down. Esses objetivos específicos indicam um compromisso com o desenvolvimento de abordagens educacionais eficazes e adaptadas às necessidades das crianças com Síndrome de Down.


\section{Contextualização}
No âmbito do ensino de matemática básica para crianças com Síndrome de Down, surge a necessidade de estratégias educacionais eficazes que não apenas transmitam conceitos matemáticos, mas também promovam o desenvolvimento de habilidades de .... A gamificação, que utiliza elementos de jogos para engajar e motivar os alunos, tem se destacado como uma abordagem eficaz para o ensino. A gamificação oferece oportunidades de aprendizado interativo, onde desafios, recompensas e feedbacks imediatos podem tornar a experiência educacional mais estimulante e atraente para as crianças.

A evolução das tecnologias tem facilitado significativamente a inclusão de pessoas com deficiência, proporcionando maior acessibilidade e autonomia. Novos recursos e APIs que facilitam a interconexão e integração entre plataformas, têm possibilitado o desenvolvimento de aplicações e soluções tecnológicas que atendam a um número cada vez maior de usuários com necessidades especiais. Um exemplo disso é o aplicativo "Hand Talk", que disponibiliza um plugin para integrações de acessibilidade a deficientes auditivos, permitindo tradução automática para Língua Brasileira de Sinais. Outras iniciativas importantes são ferramentas de leitura de tela, ampliação de conteúdo, contraste elevado e interfaces adaptáveis. Essas inovações têm o poder de reduzir barreiras, promover inclusão digital e transformar positivamente a vida de pessoas com deficiência. Como destaca  \cite{furlan2016desenvolvimento}, "a tecnologia tem um papel crucial na promoção da acessibilidade e inclusão das pessoas com deficiência".


\section{Objetivos}

Este trabalho baseia nas lições aprendidas com o trabalho \cite{2023gamificacao} anteriormente publicado, buscando aprimorar ainda mais a abordagem de gamificação no ensino de matemática para crianças com Síndrome de Down. Isso demonstra um compromisso contínuo com a pesquisa e o desenvolvimento de soluções educacionais inclusivas e eficazes.

Com base nesses pilares fundamentais, avançaremos para o desenvolvimento e implementação do aplicativo gamificado, detalhando nossos objetivos específicos e as etapas necessárias para alcançá-los. Acreditamos firmemente que este projeto contribuirá de maneira significativa para a educação inclusiva e para o bem-estar geral das crianças com Síndrome de Down, capacitando-as a enfrentar desafios acadêmicos e fortalecer suas habilidades de ..... em sua jornada rumo a uma vida mais independente e confiante.

O objetivo deste trabalho é desenvolver um aplicativo gamificado destinado ao ensino de matemática básica para crianças com Síndrome de Down, com idades entre sete e quinze anos.

\subsection{Objetivo Geral}

 Esse aplicativo tem como objetivo contribuir para aprimorar as habilidades matemáticas das crianças, ao mesmo tempo em que desempenha um papel fundamental no desenvolvimento da dicção, articulação e pronúncia. O propósito final é promover uma transformação significativa na vida cotidiana dessas crianças, aumentando sua independência, confiança e competências. O projeto visa criar uma ferramenta educacional inclusiva e estimulante, que não apenas facilite o aprendizado da matemática, mas também contribua para o aprimoramento da interação social das crianças com Síndrome de Down.

\subsection{Objetivos Específicos}

\begin{itemize}
\item{Desenvolver o aplicativo gamificado: Criar um aplicativo interativo e envolvente que seja adequado para crianças com Síndrome de Down e que se concentre no ensino de conceitos de matemática básica.}
\item {Projetar sessões e desafios educacionais: Estruturar o aplicativo em sessões temáticas com desafios e exercícios específicos de matemática.} 
\item {Integrar recompensas e feedbacks imediatos: Implementar um sistema de recompensas para incentivar o progresso e fornecer feedbacks imediatos após cada resposta e ao fim de cada sessão, para manter as crianças motivadas e engajadas no aprendizado. }  
\item {Promover a independência e confiança: Facilitar a autonomia das crianças, permitindo que elas aprendam e pratiquem matemática de forma independente, aumentando sua autoconfiança no processo.}
\item {Metrificar o rendimento gradativo do usuário: Avaliar o impacto do aplicativo no desenvolvimento cognitivo e nas habilidades matemáticas, bem como em sua interação social e independência no dia a dia.}
\item {Criar uma ferramenta inclusiva e estimulante: Assegurar que o aplicativo seja acessível e inclusivo, de forma a proporcionar uma experiência educacional enriquecedora para crianças com Síndrome de Down.}
\end{itemize}

\section{Estrutura do Artigo}

No Capítulo 1 é apresentado o objetivo geral e objetivos específicos, enquanto que o Capítulo 2 traz a Fundamentação Teórica. No Capítulo 3 está apresentada a metodologia do trabalho, enquanto que no Capítulo 4 são apresentados o experimento e coleta de dados. Finalmente, no Capítulo 5 é apresentada a conclusão desta monografia.


