\chapter{Conclusões}\label{chp:conc}

Concluímos que a evolução do aplicação ocorreu de forma fluída e assertiva, captando os principais pontos que faltavam na aplicação como a perssistencia de dados, login do usuário, alimentar mais sessões para progressão de fases. 

O desenvolvimento deste aplicativo para o ensino de matemática de forma lúdica e acessível para crianças com síndrome de Down revelou pontos positivos e oportunidades de melhoria. A abordagem em formato de jogo se mostrou engajadora, permitindo praticar os conceitos de forma motivadora. A incorporação de funcionalidades como reconhecimento de voz também trouxe benefícios de independência e confiança.

Porém, ficou evidente que alguns ajustes são necessários. É preciso expandir o conteúdo abordado e níveis de dificuldade, para cobrir de forma mais completa o currículo matemático adaptado. A adaptação da linguagem e representação visual aos diferentes estágios cognitivos do público-alvo também é essencial. E a gamificação poderia ser aprimorada com mecânicas mais sólidas de progressão e recompensa.

No geral, o projeto mostrou o potencial da tecnologia para promover o ensino inclusivo e personalizado para crianças com necessidades especiais. Mas requer continuidade, com base no aprendizado obtido nesta primeira versão. A equipe está motivada a seguir evoluindo a solução e explorando novas possibilidades.


Em conclusão, este projeto representa apenas o primeiro passo de uma trajetória. Há ainda muito trabalho para que a tecnologia educacional seja de fato inclusiva e acesse todo o seu potencial transformador. Mas com a continuidade certa, temos a convicção de estar no caminho para tornar o ensino mais igualitário e efetivo para todos.



\section{Trabalhos futuros}

Como trabalhos futuros deixaremos como sugestiva o aprimoramento da aplicação, utilizando sessões e a implementação de uma inteligência artificial para o nívelamente mais assertivo e nível de dificuldade para a criança que interagir com o aplicativo. 



