

\chapter{Metodologia}\label{chp:met}

Neste capítulo serão apresentados os procedimentos metodológicos utilizados nesta pesquisa.

\section{Objetivos da pesquisa}
Os objetivos dessa pesquisa busca avaliar a viabilidade, usabilidade e eficácia da abordagem gamificada com interação por voz, para apoiar o ensino inclusivo de matemática para crianças com síndrome de down. A pesquisa adota uma abordagem qualitativa, que se concentra na compreensão profunda e contextualizada de fenômenos sociais ou humanos. Os objetivos do estudo são exploratórios, buscando entender as experiências e percepções de crianças com síndrome de down, seus pais ou cuidadores e educadores em relação ao uso de elementos de gamificação no ensino de matemática.

A pesquisa tem o potencial de contribuir para o desenvolvimento de estratégias educacionais mais eficazes para crianças com síndrome de down. Os resultados da pesquisa poderão informar educadores, pais e pesquisadores sobre como adaptar e melhorar as estratégias educacionais para atender às necessidades específicas desse grupo de alunos.

\begin{itemize}
    \item \textbf{Natureza de Pesquisa}
    
    Esta pesquisa adota uma abordagem aplicada, o que significa que visa contribuir para a prática educacional, especialmente no contexto de crianças com necessidades especiais. O caráter qualitativo da pesquisa permite uma exploração minuciosa das vozes e experiências dos participantes.
    \item \textbf{Abordagem do Trabalho}
    
    Esta pesquisa busca avaliar a viabilidade, usabilidade e eficácia da abordagem gamificada com interação por voz, para apoiar o ensino inclusivo de matemática para crianças com Sindrome de Down. Com isso, decidimos adotar uma abordagem qualitativa. A pesquisa qualitativa é apropriada porque busca compreender as experiências, percepções e opiniões das crianças com síndrome de Down, bem como dos educadores e pais envolvidos, em relação ao uso de elementos de gamificação no ensino de matemática.

Em metodologia de pesquisa, a abordagem qualitativa se concentra na compreensão profunda e contextualizada de fenômenos sociais ou humanos. Em vez de quantificar dados numéricos, como na pesquisa quantitativa, a abordagem qualitativa procura explorar significados, percepções, experiências e contextos por meio de métodos como entrevistas, observações participantes e análise de conteúdo. Segundo \cite{creswell2013research}, a pesquisa qualitativa "envolve uma investigação aprofundada de fenômenos complexos, situados em um contexto natural, com o objetivo de descrever e interpretar as experiências das pessoas envolvidas". 
\end{itemize}

\section{Procedimentos}

\begin{itemize}
    \item \textbf{População e Amostra}
    
    \begin{itemize}
    
        \item \textbf{População Alvo}
    
        Crianças com síndrome de Down, seus pais ou cuidadores e educadores.

        \item \textbf{Critérios de Inclusão}

        Crianças com síndrome de Down que participam de programas educacionais inclusivos.

        \item \textbf{Critérios de Inclusão}

        Crianças com outras condições de desenvolvimento que não síndrome de Down.

        \item \textbf{Seleção e Amostra}

    A amostra será selecionada de forma proposital, considerando a diversidade de idades, níveis de habilidade e diferentes contextos educacionais. O tamanho da amostra será determinado por critérios de saturação, ou seja, quando não houver mais novas informações ou temas emergentes nas entrevistas.
    \end{itemize}
    \item \textbf{Instrumentos de Coleta de Dados}

    \item \textbf{Análise de Dados}

    \item \textbf{Considerações Éticas}
    
\end{itemize}

\section{Discussão Esperada}

O ensino de matemática para pessoas com síndrome de Down apresenta desafios únicos. Devido às dificuldades cognitivas associadas à síndrome, como problemas com memória, atenção e raciocínio abstrato, é preciso uma abordagem diferenciada que considere as necessidades individuais de cada aluno.

Métodos multissensoriais, com uso de imagens, manipulação de objetos e situações práticas podem favorecer a compreensão dos conceitos matemáticos. O ritmo mais lento de aprendizagem também deve ser levado em conta, dando mais tempo e repetição das atividades. O reforço positivo e a valorização dos pequenos progressos ajudam a manter a motivação. Adaptar o currículo, priorizando habilidades funcionais também é essencial. 

O uso de tecnologias como calculadoras e softwares educativos também pode ser útil para facilitar o aprendizado de matemática por pessoas com síndrome de Down. As calculadoras ajudam a minimizar erros de cálculo, permitindo que os alunos foquem nos conceitos em vez dos aspectos computacionais. Já softwares com interface visual e interativa podem tornar noções abstratas mais concretas e intuitivas. Porém, é importante dosar o uso da tecnologia para que os alunos também pratiquem o cálculo manual e desenvolvam habilidades básicas.

Além dos desafios cognitivos, questões socioemocionais também precisam ser consideradas no ensino de matemática para alunos com síndrome de Down. Observa-se frequentemente que eles tem baixa autoestima e se frustram diante de erros. Portanto, é essencial promover um ambiente seguro, paciente e motivador, que encoraje a exploração e legitime tentativas e falhas como parte do processo de aprendizagem. Isso ajuda os alunos a desenvolverem autoconfiança e persistência para enfrentar os obstáculos matemáticos de forma resiliente.

Espera-se que esta pesquisa forneça uma compreensão aprofundada das implicações da gamificação no ensino de matemática para crianças com síndrome de Down. Os resultados deverão informar educadores, pais e pesquisadores sobre como adaptar e melhorar as estratégias educacionais para atender às necessidades específicas desse grupo de alunos.






