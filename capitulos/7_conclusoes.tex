\chapter{Considerações Finais}\label{chp:conc}

O projeto Math.Pow foi desenvolvido com o objetivo de criar uma ferramenta educacional gamificada que auxilie no ensino de matemática básica, com foco inicial nos alunos do Ensino Fundamental I . O aplicativo busca tornar o aprendizado mais interativo e motivador, utilizando elementos de jogos para engajar os alunos e facilitar a compreensão dos conceitos matemáticos.

A análise das matrizes curriculares do Ensino Fundamental I, especialmente para o 4º ano, revelou a necessidade de uma abordagem mais integrada e contínua no ensino de matemática. A comparação entre o conteúdo programático do Colégio Salesiano do Salvador e o Referencial Curricular Municipal de Salvador destacou diferenças significativas em suas abordagens e detalhamentos. A partir dessa análise, foram extraídas oito unidades temáticas principais: operações básicas, frações, álgebra, geometria, grandezas e medidas, probabilidade e estatística, educação financeira e algarismos romanos.

Durante o desenvolvimento do Math.Pow, a metodologia adotada priorizou a interatividade e a personalização do aprendizado. O aplicativo foi projetado para se adaptar ao ritmo e ao estilo de aprendizado de cada aluno, oferecendo feedback imediato e recompensas para manter o engajamento. A validação com professores foi um marco importante para refinar a aplicação, garantindo que ela atendesse às necessidades reais do ambiente escolar.

Com base nos resultados obtidos por meio da aplicação do questionário e da análise das percepções de pais e professores, o aplicativo Math.Pow demonstrou grande potencial. A pesquisa forneceu insights valiosos, destacando os benefícios e as oportunidades de melhoria do aplicativo, reforçando seu papel como um recurso complementar eficaz.

A aceitação geral da ferramenta foi positiva, tanto por parte dos pais quanto dos professores, evidenciando que a gamificação pode ser uma estratégia eficiente para estimular o interesse e o engajamento das crianças no aprendizado de matemática. Cerca de 70\% dos pais acreditam que o aplicativo tem o potencial de beneficiar significativamente o aprendizado, e 60\% dos professores reconhecem que os elementos gamificados aumentam a autoestima e confiança das crianças na disciplina.

Entre os aspectos positivos identificados, destacaram-se o sistema de pontuação e recompensas, a interface intuitiva e as atividades que respeitam os níveis de desenvolvimento das crianças. Crianças mais velhas, de 9 a 10 anos, apresentaram maior autonomia no uso do aplicativo, enquanto as mais novas, de 6 a 7 anos, necessitaram de maior suporte parental, evidenciando a necessidade de adaptar o conteúdo e funcionalidades às diferentes faixas etárias.

Apesar dos resultados promissores, a pesquisa também revelou desafios a serem enfrentados. Do ponto de vista dos pais, questões como maior suporte para iniciantes, inclusão de elementos interativos e adaptação do conteúdo para diferentes níveis de dificuldade foram apontadas como áreas de melhoria. Professores, por sua vez, levantaram preocupações sobre a acessibilidade do aplicativo em escolas públicas, especialmente pela dependência de conexão com a internet e a limitação de recursos tecnológicos em determinados contextos educacionais.

Dessa forma, o Math.Pow demonstrou-se promissor ao abordar dificuldades comuns no ensino de matemática, como falta de interesse, dificuldade de concentração e barreiras na aprendizagem. Ao mesmo tempo, oferece possibilidades para tornar o aprendizado mais lúdico, motivador e eficaz.

Por fim, a continuidade no uso de ferramentas gamificadas como o Math.Pow tem o potencial de transformar o ensino de matemática, tornando-o mais acessível, envolvente e eficaz. As melhorias propostas, alinhadas ao feedback coletado, reforçam o compromisso com a evolução da ferramenta e seu impacto no aprendizado das crianças. Com ajustes pontuais e atenção às necessidades específicas dos usuários, o aplicativo poderá consolidar-se como uma solução educacional inovadora e indispensável.

\section{Conclusões e Recomendações Baseadas na Pesquisa com Professores}

A pesquisa realizada com professores reforçou a percepção positiva do potencial do aplicativo. Os educadores destacaram que a gamificação não apenas estimula o interesse e a motivação dos alunos, mas também contribui para um aprendizado mais eficaz e dinâmico. Além disso, funcionalidades como reconhecimento de voz foram reconhecidas como recursos que aumentam a independência e a confiança das crianças ao interagir com a aplicação.

A pesquisa também evidenciou discrepâncias quanto à aceitação tecnológica, com professores mais jovens mostrando maior entusiasmo em relação à gamificação, enquanto os mais experientes expressaram certo ceticismo. No entanto, mesmo aqueles que se mostraram mais neutros ou céticos reconheceram que o aplicativo pode servir como uma ferramenta de suporte, ainda que com limitações.

Desafios foram identificados, especialmente no que tange à acessibilidade em escolas públicas com recursos limitados. A dependência de conexão com a internet foi apontada como um obstáculo significativo. Para enfrentar esse problema, recomenda-se o desenvolvimento de funcionalidades offline, ampliando o alcance e a usabilidade do Math.Pow.

Adicionalmente, observou-se a necessidade de aprimorar o design e a interface do aplicativo, tornando-os mais lúdicos e visualmente atraentes para as diferentes faixas etárias do público-alvo. Outra recomendação importante envolve a expansão do conteúdo e dos níveis de dificuldade, garantindo uma maior cobertura do currículo matemático adaptado. Também é essencial ajustar a linguagem e a representação visual aos estágios cognitivos das crianças, promovendo uma experiência inclusiva e eficaz para todos os usuários.

\section{Trabalhos Futuros}

Como próximos passos, sugere-se o aprimoramento contínuo da aplicação. Isso inclui a implementação de mecanismos mais robustos para nivelamento adaptativo, ajustando o nível de dificuldade automaticamente de acordo com o progresso do aluno. Além disso, explorar novas funcionalidades gamificadas, como uma loja de itens, pode aumentar ainda mais o engajamento e a motivação dos alunos.

Com base em nossa pesquisa com pais e professores, definimos os seguintes aprimoramentos a serem implementados como melhorias futuras:

\begin{itemize}
    \item Aprimorar a acessibilidade: Desenvolver funcionalidades offline para atender usuários com limitações de conectividade.
    \item Personalizar as experiências: Implementar níveis de dificuldade adaptáveis e conteúdo customizado para diferentes perfis de alunos.
    \item Expandir os elementos lúdicos: Adicionar personagens interativos e jogos mais dinâmicos para aumentar o engajamento.
    \item Capacitar professores e pais: Oferecer guias de uso e orientações para facilitar a implementação em diferentes contextos educacionais.
\end{itemize}
    
A integração de feedback dos usuários e educadores será crucial para orientar as próximas fases do desenvolvimento. A incorporação de elementos mais dinâmicos, como personagens interativos e histórias envolventes, também pode contribuir para uma experiência de aprendizado ainda mais cativante.

Apesar de ainda haver muito a ser aprimorado, o projeto mostrou-se promissor e está no caminho certo para transformar o aprendizado de matemática em uma experiência significativa e acessível para crianças em diferentes contextos educacionais.

