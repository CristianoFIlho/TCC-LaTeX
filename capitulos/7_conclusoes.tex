\chapter{Conclusões}\label{chp:conc}

\subsection{Síntese} 

O projeto Math.Pow foi desenvolvido com o objetivo de criar uma ferramenta educacional gamificada para o ensino de matemática básica a alunos do Ensino Fundamental I. A aplicação visa tornar o aprendizado mais interativo e motivador, utilizando elementos de jogos para engajar os alunos e facilitar a compreensão dos conceitos matemáticos.

A análise das matrizes curriculares do Ensino Fundamental I, especificamente para o 4º ano, revelou a necessidade de uma abordagem mais integrada e contínua no ensino de matemática. A comparação entre o conteúdo programático do Colégio Salesiano do Salvador e o Referencial Curricular Municipal de Salvador destacou diferenças significativas em suas abordagens e detalhamentos. A partir dessa análise, foram extraídas oito unidades temáticas principais: operações básicas, frações, álgebra, geometria, grandezas e medidas, probabilidade e estatística, educação financeira e algarismos romanos.

O desenvolvimento do Math.Pow seguiu uma metodologia que priorizou a interatividade e a personalização do aprendizado. A aplicação foi projetada para adaptar-se ao ritmo e estilo de aprendizado de cada aluno, oferecendo feedback imediato e recompensas para manter o engajamento. A validação com professores do Ensino Fundamental I foi crucial para refinar a aplicação, garantindo que ela atendesse às necessidades reais do ambiente escolar.

\subsection{Conclusão}

Concluímos que a evolução da aplicação Math.Pow ocorreu de forma fluida e assertiva, captando os principais pontos que faltavam na aplicação, como a persistência de dados, login do usuário e a inclusão de mais sessões para progressão de fases. O desenvolvimento desta idealização e estudo de negócio para o ensino de matemática de forma lúdica e acessível para crianças revelou pontos positivos e oportunidades de melhoria.

A abordagem em formato de jogo se mostrou engajadora, permitindo praticar os conceitos de forma motivadora. A incorporação de funcionalidades como reconhecimento de voz também trouxe benefícios de independência e confiança. No entanto, ficou evidente que alguns ajustes são necessários. É preciso expandir o conteúdo abordado e os níveis de dificuldade para cobrir de forma mais completa o currículo matemático adaptado. A adaptação da linguagem e representação visual aos diferentes estágios cognitivos do público-alvo também é essencial. A gamificação poderia ser aprimorada com mecânicas mais sólidas de progressão e recompensa.

No geral, o projeto mostrou o potencial da tecnologia para promover o ensino inclusivo e personalizado para crianças. Mas requer continuidade, com base no aprendizado obtido nesta primeira versão. A equipe está motivada a seguir evoluindo a solução e explorando novas possibilidades.

Em conclusão, este projeto representa apenas o primeiro passo de uma trajetória. Há ainda muito trabalho para que a tecnologia educacional seja de fato inclusiva e acesse todo o seu potencial transformador. Mas com a continuidade certa, temos a convicção de estar no caminho para tornar o ensino mais igualitário e efetivo para todos.

\section{Trabalhos Futuros}

Como trabalhos futuros, sugerimos o aprimoramento da aplicação, utilizando sessões e a implementação de uma funcionalidade para o nivelamento mais assertivo e o ajuste do nível de dificuldade para a criança que interagir com o aplicativo. Além disso, a continuidade da gamificação com a loja de itens pode ser explorada para aumentar ainda mais o engajamento dos alunos.


Essa síntese e conclusão refletem o desenvolvimento e os resultados do projeto Math.Pow, destacando os pontos fortes e as áreas de melhoria, além de sugerir direções futuras para o aprimoramento da aplicação.