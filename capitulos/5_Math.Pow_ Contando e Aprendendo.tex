\chapter{Contando e Aprendendo}\label{chp:mat}

\section{Justificativa e Significado do Nome da Aplicação: } 

O nome \textit{Math.Pow} foi cuidadosamente escolhido para a aplicação de matemática básica, em referência direta à função \texttt{Math.pow()} da linguagem de programação JavaScript, que serve como base tecnológica para o desenvolvimento do projeto. Essa denominação reflete não apenas a fundamentação técnica da aplicação, mas também sua missão educacional de potencializar o aprendizado em matemática.

\subsection{A Função \text{Math.pow()} em JavaScript}

Na programação em JavaScript, a função \texttt{Math.pow()} pertence ao objeto nativo \texttt{Math} e é utilizada para realizar operações de exponenciação.A função \texttt{Math.pow()} é fundamental em aplicações que requerem cálculos matemáticos precisos, desempenhando um papel crucial em algoritmos que envolvem potenciação. Internamente, essa função é implementada utilizando algoritmos eficientes, como a exponenciação rápida, permitindo o cálculo de potências com alto desempenho, mesmo para números elevados.

\subsection{Motivação para o Nome da Aplicação}

A escolha do nome \textit{Math.Pow} para a aplicação justifica-se por diversos fatores que combinam aspectos técnicos e simbólicos:

\begin{enumerate}
    \item \textbf{Alinhamento Tecnológico}: O nome reflete a linguagem de programação empregada no desenvolvimento (JavaScript/TypeScript), destacando a conexão com tecnologias web modernas e evidenciando a competência técnica da equipe.

    \item \textbf{Metáfora Educacional}: A palavra "pow", derivada de "power" (poder/potência), simboliza o empoderamento e o crescimento no aprendizado matemático dos usuários, assim como a função \texttt{Math.pow()} eleva números a potências superiores.

    \item \textbf{Integração entre Tecnologia e Educação}: A referência direta à função matemática cria uma ponte entre o mundo da programação e o ensino da matemática básica, representando a interdisciplinaridade que fundamenta o projeto.

    \item \textbf{Facilidade de Reconhecimento}: Utilizar um termo conhecido na programação facilita a memorização e cria uma identidade única para a aplicação, especialmente entre estudantes e entusiastas de tecnologia.

    \item \textbf{Inovação e Modernidade}: O nome sugere uma abordagem inovadora e tecnológica para o ensino da matemática, alinhando-se com as tendências atuais de educação digital e gamificação.
\end{enumerate}

\subsection{Potencializando o Aprendizado}

Assim como a função \texttt{Math.pow()} amplifica um número ao elevá-lo a uma potência, a aplicação \textit{Math.Pow} tem como objetivo potencializar o conhecimento dos estudantes, oferecendo recursos que aprimoram o aprendizado de matemática básica por meio de:

\begin{itemize}
    \item \textbf{Interatividade}: Implementação de interfaces intuitivas e interativas que facilitam a compreensão dos conceitos matemáticos de forma envolvente.

    \item \textbf{Gamificação}: Incorporação de mecânicas de jogo para aumentar o engajamento e a motivação dos usuários, tornando o aprendizado mais prazeroso e eficaz.

    \item \textbf{Personalização}: Adaptação dos conteúdos e desafios ao nível de conhecimento.

    \item \textbf{Eficiência Tecnológica}: Utilização de algoritmos e técnicas de programação que otimizam o desempenho da aplicação, garantindo uma experiência de uso fluida e responsiva.
\end{itemize}


A denominação \textit{Math.Pow} transcende um simples título; ela simboliza os objetivos e valores da aplicação. Ao conectar a função matemática em JavaScript com a missão educacional, a aplicação busca ser uma ferramenta que impulsiona o aprendizado, eleva o potencial dos estudantes e integra a tecnologia de forma significativa no processo educativo. Assim como a função \texttt{Math.pow()} é essencial para operações matemáticas avançadas na programação, a aplicação \textit{Math.Pow} pretende ser essencial no desenvolvimento das habilidades matemáticas fundamentais de seus usuários.


\subsection{Paleta de Cores, Elementos de Design e Topografia das cores}

\subsubsection{Paleta de Cores}

A aplicação faz uso de uma paleta de cores equilibrada, combinando tons suaves com acentos vibrantes para criar uma experiência visual agradável e focada na educação matemática. As cores principais incluem:

\begin{itemize}
    \item \textbf{Fundo Principal (mainBackground)}: \#757FB2 - Um tom de azul claro que proporciona um ambiente calmo e acolhedor.
    \item \textbf{Contraste de Fundo (backgroundContrast)}: \#AAB2D5 - Utilizado para diferenciar seções e criar hierarquia visual.
    \item \textbf{Branco (white)}: \#EFF1FA - Usado em fundos de componentes e textos para garantir legibilidade.
    \item \textbf{Preto (black)}: \#4B4B4B - Empregado em textos primários, oferecendo contraste forte com fundos claros.
    \item \textbf{Preto Desbotado (fadedBlack)}: \#8D8D91 - Para textos secundários e placeholders, adicionando sutileza.
    \item \textbf{Acento (accent)}: \#FFF8C9 - Um amarelo suave para destacar elementos importantes ou interativos.
    \item \textbf{Sucesso (green)}: \#50FA7B - Indica ações positivas ou respostas corretas.
    \item \textbf{Erro (red)}: \#FF5555 - Sinaliza erros ou respostas incorretas, atraindo atenção imediata.
\end{itemize}

Essa paleta equilibra tons frios com acentos quentes, facilitando a concentração e mantendo o interesse do usuário.


\subsubsection{Termos Técnicos e Elementos}

\begin{itemize}
    \item \textbf{Componentes Reutilizáveis}: A aplicação utiliza componentes como \texttt{Button}, \texttt{TextInput}, \texttt{UserInfoHeader} e \texttt{SessionCard} para manter consistência na interface.
    \item \textbf{Tipografia}: Uso da fonte personalizada \texttt{"Righteous"} para títulos e cabeçalhos, reforçando a identidade visual.
    \item \textbf{Ícones e Imagens}: Integração de ícones das bibliotecas \texttt{MaterialIcons}, \texttt{FontAwesome5} e imagens personalizadas (e.g., \texttt{squid.png}) enriquecem a interface e facilitam a navegação.
    \item \textbf{Animações}: Animações \texttt{Lottie} são incorporadas para fornecer feedback interativo em eventos de sucesso ou falha nas sessões.
    \item \textbf{Layouts Responsivos}: Utilização de utilitários como \texttt{SCREEN\_WIDTH}, \texttt{SCREEN\_HEIGHT} e \texttt{actuatedNormalize} para garantir compatibilidade em diferentes tamanhos de tela.
\end{itemize}

O design prioriza a usabilidade, acessibilidade e engajamento, proporcionando uma experiência intuitiva e agradável para os usuários no aprendizado matemático.

\subsection{Estrutura de Navegação e Design da Aplicação}

\subsubsection{Estrutura de Navegação}

A navegação na aplicação \textit{Math Pow} é delineada de forma modular e intuitiva, garantindo uma experiência fluida para o usuário. Conforme ilustrado no Diagrama de Casos de Uso UML, os principais atores envolvem o Usuário e o Sistema, com casos de uso que abrangem desde a autenticação até a interação com cursos e sessões de aprendizado. A autenticação é centralizada, permitindo que usuários registrem contas, façam login, recuperem senhas e realizem logout de maneira eficiente.

Após a autenticação, a estrutura de navegação se expande para incluir a visualização e seleção de cursos, iniciando sessões de estudo conforme demonstrado no Diagrama de Classes UML. A aplicação utiliza o \textit{Expo Router} para gerenciar rotas baseadas em arquivos, organizando as telas em módulos públicos e autenticados dentro da pasta \texttt{modules}. Esta abordagem modular facilita a manutenção e a escalabilidade do sistema, permitindo a adição de novas funcionalidades com impacto mínimo na estrutura existente.

\subsubsection{Propriedades de Navegabilidade}

A navegabilidade é suportada por uma arquitetura robusta, onde componentes reutilizáveis desempenham um papel crucial na consistência da interface. Componentes como \texttt{CourseList}, \texttt{StatusCard} e \texttt{UserInfoHeader} são empregados em várias partes da aplicação, promovendo uma experiência homogênea e reduzindo redundâncias no código. O Diagrama de Sequência UML detalha as interações entre diferentes componentes durante processos críticos, como a autenticação e a iniciação de uma sessão de estudo.

A utilização de navegação por abas, implementada no componente \texttt{CustomTabBar}, permite que os usuários alternem facilmente entre diferentes seções da aplicação, como cursos, perfil e conquistas. Este design facilita o acesso rápido às funcionalidades principais, melhorando a usabilidade e a satisfação do usuário.

\subsubsection{Elementos de Design e Paleta de Cores}

Do ponto de vista de design, a aplicação adota uma paleta de cores equilibrada que combina tons suaves com acentos vibrantes, conforme detalhado na análise anterior. A utilização de cores como azul claro (\#757FB2) para fundos principais e amarelo suave (\#FFF8C9) para elementos de destaque cria um ambiente visualmente agradável e focado na educação matemática. Essa escolha cromática não apenas melhora a estética, mas também contribui para a funcionalidade, facilitando a distinção de elementos interativos e informações importantes.

Os componentes reutilizáveis são estilizados utilizando \textit{Styled Components} e \textit{Restyle}, garantindo consistência visual e adaptabilidade a diferentes tamanhos de tela. A tipografia, com fontes como \texttt{"Righteous"} para títulos, reforça a identidade visual da aplicação, enquanto ícones provenientes de bibliotecas como \texttt{MaterialIcons} e \texttt{FontAwesome5} enriquecem a interface, tornando-a mais intuitiva e de fácil navegação.

\subsubsection{Fluxo de Usuário e Experiência Interativa}

O Diagrama de Atividade UML mapeia o fluxo principal de interação do usuário, desde o login até a finalização de uma sessão de estudo. O processo começa com a autenticação, seguido pela seleção de cursos e iniciação de sessões, onde o usuário responde a questões, acumula pontos e gerencia vidas. Este fluxo é projetado para ser linear e lógico, evitando complexidades desnecessárias e mantendo o usuário engajado.

Além disso, o Diagrama de Sequência UML descreve em detalhes as interações entre componentes durante a autenticação e a execução de sessões. Por exemplo, ao inserir credenciais na tela de login, o sistema interage com o \texttt{AuthStore} e o \texttt{Firebase} para verificar e autenticar o usuário, garantindo segurança e eficiência no processo.

\subsubsection{Componentização e Reutilização}

A aplicação enfatiza a componentização, utilizando uma vasta gama de componentes reutilizáveis localizados em \texttt{components}. Componentes como \texttt{FaqCard}, \texttt{CourseCard} e \texttt{Button} são projetados para serem modulares e independentes, facilitando a manutenção e a extensão da aplicação. Essa abordagem modular também contribui para a consistência visual e funcional, permitindo que alterações em um componente sejam automaticamente refletidas em todas as suas instâncias.

A aplicação \textit{Math Pow} está arquitetada de maneira a proporcionar uma experiência de usuário fluida e envolvente, suportada por uma estrutura de navegação bem definida e elementos de design cuidadosamente selecionados. Os diagramas UML presentes na pasta \texttt{UML Diagrams} oferecem uma visão detalhada e abrangente da interação entre componentes, fluxos de atividades e casos de uso, embasando as escolhas de design e implementações técnicas realizadas ao longo do desenvolvimento.

Essa síntese evidencia a importância de uma abordagem modular e centrada no usuário no desenvolvimento de aplicações educacionais, assegurando não apenas a funcionalidade, mas também a usabilidade e a estética, elementos fundamentais para o sucesso e adoção da aplicação \textit{Math Pow} pelos seus usuários.