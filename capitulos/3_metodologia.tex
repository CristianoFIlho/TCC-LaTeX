

\chapter{Metodologia}\label{chp:met}


\section{Objetivos da pesquisa}
Os objetivos dessa pesquisa busca avaliar a viabilidade, usabilidade e eficácia da abordagem gamificada como ferramenta de auxílio educacional, com base em aplicações educacionais já validades e concretizadas para apoiar o ensino inclusivo de matemática para crianças com enfoque inicialmente do público alvo do ensido fundamental I. A pesquisa adota uma abordagem qualitativa, que se concentra na compreensão profunda e contextualizada de fenômenos sociais ou humanos. Os objetivos do estudo são exploratórios, buscando entender as experiências e percepções de crianças e suas maneiras de interaçao com tecnologias e aplicações de ensino, seus pais ou cuidadores e educadores em relação ao uso de elementos de gamificação no ensino de matemática.

A pesquisa tem o potencial de contribuir para o desenvolvimento de estratégias educacionais mais eficazes para crianças em um âmbito específico. Os resultados da pesquisa poderão informar educadores, pais e pesquisadores sobre como adaptar e melhorar as estratégias educacionais para atender às necessidades específicas desse grupo de alunos.

\begin{itemize}
    \item \textbf{Natureza de Pesquisa}
    
    Esta pesquisa adota uma abordagem aplicada, o que significa que visa contribuir para a prática educacional, especialmente no contexto de crianças com dificuldades em metodologias tradicionais. O caráter qualitativo da pesquisa permite uma exploração do entendimento de forma lúdica e contemporanea do ensino usando a tecnologia como ferramenta central e de apoio ao docente.
    \item \textbf{Abordagem do Trabalho}
    
    Esta pesquisa busca avaliar a viabilidade, usabilidade e eficácia da abordagem gamificada, para apoiar o ensino inclusivo de matemática para crianças. Com isso, decidimos adotar uma abordagem qualitativa. A pesquisa qualitativa é apropriada porque busca compreender as experiências, percepções e opiniões das crianças, bem como dos educadores e pais envolvidos, em relação ao uso de elementos de gamificação no ensino de matemática.

Em metodologia de pesquisa, a abordagem qualitativa se concentra na compreensão profunda e contextualizada de fenômenos sociais ou humanos. Em vez de quantificar dados numéricos, como na pesquisa quantitativa, a abordagem qualitativa procura explorar significados, percepções, experiências e contextos por meio de métodos como entrevistas, observações participantes e análise de conteúdo. Segundo \cite{creswell2013research}, a pesquisa qualitativa "envolve uma investigação aprofundada de fenômenos complexos, situados em um contexto natural, com o objetivo de descrever e interpretar as experiências das pessoas envolvidas". 
\end{itemize}

\section{Procedimentos}

\begin{itemize}
    \item \textbf{População e Amostra}
    
    \begin{itemize}
    
        \item \textbf{População Alvo}
    
        Crianças do ensino fundamental I.

        \item \textbf{Critérios de Inclusão}

        Crianças da turma do 4º ano do ensino fundamental I que participam da pesquisa de validação da aplicação.


        \item \textbf{Seleção e Amostra}

    A amostra será selecionada de forma proposital, considerando a diversidade de idades, níveis de habilidade e diferentes contextos educacionais. O tamanho da amostra será determinado por critérios de saturação, ou seja, quando não houver mais novas informações ou temas emergentes nas entrevistas.
    \end{itemize}

    
\end{itemize}

\section{Discussão Esperada}

\subsection{Ensino de Matemática para Crianças do Ensino Fundamental I}

O ensino de matemática para crianças do ensino fundamental I apresenta desafios únicos. Devido às dificuldades relacionadas à atenção e ao foco, como problemas com memória, atenção e raciocínio abstrato, é preciso uma abordagem diferenciada que considere as necessidades individuais de cada aluno.

\begin{itemize}
    \item \textbf{Métodos multissensoriais}

Métodos multissensoriais, com uso de imagens, manipulação de objetos e situações práticas podem favorecer a compreensão dos conceitos matemáticos. O ritmo mais lento de aprendizagem também deve ser levado em conta, dando mais tempo e repetição das atividades. O reforço positivo e a valorização dos pequenos progressos ajudam a manter a motivação. Adaptar o currículo, priorizando habilidades funcionais também é essencial. 

 \item \textbf{Uso de Tecnologias}

 Tecnologias como calculadoras e softwares educativos podem facilitar o aprendizado de matemática. As calculadoras ajudam a minimizar erros de cálculo, permitindo que os alunos foquem nos conceitos em vez dos cálculos. Softwares com interfaces visuais e interativas tornam noções abstratas mais concretas e intuitivas. No entanto, é importante dosar o uso da tecnologia para que os alunos também pratiquem o cálculo manual e desenvolvam habilidades básicas.

\item \textbf{Ambiente de Aprendizagem}

Além dos desafios cognitivos, questões socioemocionais também precisam ser consideradas no ensino de matemática para essa faixa etária. Manter o foco e a concentração pode ser difícil em um ambiente imersivo com uso de tecnologias como smartphones e tablets. Portanto, é essencial promover um ambiente seguro, paciente e motivador. Esse ambiente deve encorajar a exploração e legitimar tentativas e falhas como parte do processo de aprendizagem. Isso ajuda os alunos a desenvolverem autoconfiança e persistência para enfrentar os desafios matemáticos de forma resiliente.

\item \textbf{Qual o enfoque da pesquisa em questão}

Espera-se que essas estratégias ajudem educadores, pais e pesquisadores a adaptar e melhorar as práticas educacionais para atender às necessidades específicas das crianças do ensino fundamental I. Promover um ambiente de aprendizado adequado e utilizar métodos eficazes são passos essenciais para o sucesso no ensino de matemática. 

\end{itemize}












