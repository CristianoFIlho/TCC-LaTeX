\chapter{Experimento e Coleta de Dados} \label{chp:exp}

\section{Análise da Matriz Curricular do Ensino Fundamental 1 (4º Ano) na Matéria de Matemática}

A análise da matriz curricular do Ensino Fundamental 1, especificamente para o 4º ano na matéria de Matemática, envolve a comparação entre o conteúdo programático do Colégio Salesiano do Salvador e o Referencial Curricular Municipal de Salvador. Ambos os documentos têm como objetivo proporcionar uma educação matemática abrangente, mas diferem em sua abordagem e detalhamento.

\subsection{Conteúdo Programático do Colégio Salesiano do Salvador}

No Colégio Salesiano do Salvador, o currículo de Matemática para o 4º ano é dividido em três trimestres, cada um abordando tópicos específicos. No primeiro trimestre, os alunos são introduzidos aos números romanos e ao sistema de numeração decimal na base 10, com foco em números até a dezena de milhar. Além disso, há uma ênfase na composição e decomposição de números, valor posicional, arredondamento e estimativa.

Durante o segundo trimestre, o currículo avança para multiplicação e divisão, introduzindo conceitos como múltiplos, propriedades da multiplicação e divisão por 10, 100 e 1000. A geometria é também abordada, com os alunos aprendendo sobre perímetro e a ideia de área. Estatística e probabilidade são introduzidas, com foco em variáveis categóricas e numéricas e na criação e interpretação de tabelas e gráficos.

No terceiro trimestre, o currículo foca em números decimais, abordando sua utilização no sistema monetário brasileiro, bem como adição e subtração de números decimais. Geometria continua a ser explorada, incluindo conceitos de ângulos (retos, agudos e obtusos), retas paralelas e perpendiculares, simetria, prismas e pirâmides. A educação financeira é introduzida, ensinando conceitos de consumo, valores de vida e empreendedorismo \cite{conteudo_programatico_salesiano}.

\subsection{Referencial Curricular Municipal de Salvador}

O Referencial Curricular Municipal para os anos iniciais do ensino fundamental de Salvador apresenta uma abordagem integrada e contínua para a aprendizagem matemática. Este currículo também abrange números e operações, geometria, grandezas e medidas, e tratamento da informação, mas com uma estrutura menos fragmentada por trimestre.

Os alunos são incentivados a resolver problemas que envolvem adição, subtração, multiplicação e divisão com números naturais, além de realizar diferentes tipos de cálculos (exatos, aproximados e mentais). A geometria é integrada ao aprendizado diário, com foco na identificação, descrição e representação da localização e movimentação de objetos no espaço, bem como na análise e comparação de figuras geométricas planas e espaciais.

Grandezas e medidas são abordadas através da resolução de problemas que envolvem a estimativa e medição de diferentes quantidades, como tempo, comprimento, massa e capacidade. Os alunos aprendem a utilizar medidas de tempo e a realizar conversões simples, além de aplicar esses conceitos em situações práticas.

O tratamento da informação é enfatizado com a coleta e interpretação de dados, apresentando resultados em tabelas e gráficos de colunas. Os alunos são ensinados a coletar dados de duas variáveis, a criar tabelas e gráficos simples e de dupla entrada, e a interpretar as informações apresentadas \cite{referencial_curricular_municipal}.

\section{Comparação entre Matrizes de Ensino: Temas e Abordagens Principais}

Comparando os dois currículos, foi observado que ambos cobrem os principais tópicos de Matemática necessários para o 4º ano, mas com diferenças notáveis em sua abordagem. O currículo do Colégio Salesiano do Salvador é mais detalhado e dividido claramente por trimestres, oferecendo uma estrutura bem definida para cada período letivo. Em contraste, o Referencial Curricular Municipal de Salvador adota uma abordagem mais integrada, com foco na aplicação prática dos conhecimentos e na resolução de problemas contínua ao longo do ano.

A inclusão de educação financeira no currículo do Colégio Salesiano é um diferencial, pois aborda aspectos como consumo, valores de vida e empreendedorismo, preparando os alunos para lidar com questões financeiras desde cedo. Por outro lado, o Referencial Curricular Municipal de Salvador enfatiza a interpretação de dados e a aplicação prática dos conceitos matemáticos, promovendo uma compreensão mais ampla e contextualizada da matemática no dia a dia dos alunos.

Ambos os currículos são eficazes em seus objetivos, oferecendo uma educação matemática completa e adequada para os alunos do 4º ano, mas com metodologias e enfoques distintos que refletem suas respectivas filosofias educacionais.

\begin{longtable}{|p{3cm}|p{6cm}|p{6cm}|}
 
\hline
\textbf{Tópicos} & \textbf{Colégio Salesiano do Salvador} & \textbf{Referencial Curricular Municipal de Salvador} \\
\hline
Números e Operações & Números romanos, sistema de numeração decimal, números até dezena de milhar, adição, subtração, multiplicação, divisão & Relações subjacentes a números, operações com números naturais, cálculos exatos, aproximados e mentais \\
\hline
Geometria & Perímetro, área, ângulos, retas paralelas e perpendiculares, prismas, pirâmides, simetria & Localização e movimentação de objetos, figuras geométricas planas e espaciais, ângulos retos e não retos \\
\hline
Estatística e Probabilidade & Variáveis categóricas e numéricas, tabelas, gráficos, gráfico pictórico & Estimativa e medição de tempo, comprimento, massa, capacidade, conversões simples \\
\hline
Grandezas e Medidas & Medidas de comprimento, massa, capacidade, tempo, temperatura & Estimativa de diferentes medidas, medições efetivas, uso de medidas de tempo \\
\hline
Educação Financeira & Consumo, valores de vida, crenças, paradigmas, aposentadoria, proteção, empreendedorismo, metas, cooperativismo, renda e consumo & Não especificado \\
\hline
Tratamento da Informação & Coleta de dados, apresentação em tabelas e gráficos & Coleta e interpretação de dados, criação de tabelas e gráficos simples e de dupla entrada \\
\hline
\caption{Tabela comparativa entre matrizes curriculares de matemática básica do ensino fundamental 1, 4º ano}
\label{tab:comparacao_matrizes}
\end{longtable}

\section{Estruturação e Acompanhamento dos Temas Abordados na Aplicação}

Para enriquecer essa análise, utilizou-se o Khan Academy como uma base de dados para coleta de assuntos, garantindo uma abordagem ampla e atualizada dos tópicos. Baseados no conteúdo programático anterior, chegamos ao filtro dos conteúdos mais relevantes para a aplicação. A plataforma fornece uma vasta gama de materiais educativos, exercícios práticos e vídeos explicativos que cobrem os tópicos abordados nas matrizes curriculares \cite{khanacademy_numeros_soma_subtracao}.

\begin{longtable}{|p{2cm}|p{4cm}|p{4cm}|p{4cm}|}
\hline
\textbf{Unidade} & \textbf{Tópicos no Khan Academy} & \textbf{Colégio Salesiano do Salvador} & \textbf{Referencial Curricular Municipal do Salvador} \\
\hline
Unidade 1 & Números: soma e subtração & Adição, subtração, propriedades da adição, expressões numéricas, situações-problema & Relações subjacentes a números, operações com números naturais, cálculos exatos, aproximados e mentais \\
\hline
Unidade 2 & Números: multiplicação e divisão & Multiplicação, divisão, propriedades da multiplicação, regularidades da divisão, situações-problema & Relações subjacentes a números, operações com números naturais, cálculos exatos, aproximados e mentais \\
\hline
Unidade 3 & Números: frações & Não especificado & Noção de fração, leitura de fração, comparação de frações, adição e subtração de frações, situações-problema \\
\hline
Unidade 4 & Álgebra & Não especificado & Noção de equação, valor desconhecido, relações de igualdade, operações inversas \\
\hline
Unidade 5 & Geometria & Perímetro, área, ângulos, retas paralelas e perpendiculares, prismas, pirâmides, simetria & Localização e movimentação de objetos, figuras geométricas planas e espaciais, ângulos retos e não retos \\
\hline
Unidade 6 & Grandezas e medidas & Medidas de comprimento, massa, capacidade, tempo, temperatura & Estimativa de diferentes medidas, medições efetivas, uso de medidas de tempo \\
\hline
Unidade 7 & Probabilidade e estatística & Variáveis categóricas e numéricas, tabelas, gráficos, gráfico pictórico & Coleta e interpretação de dados, criação de tabelas e gráficos simples e de dupla entrada \\
\hline
Unidade 8 & Educação financeira & Consumo, valores de vida, crenças, paradigmas, aposentadoria, proteção, empreendedorismo, metas, cooperativismo, renda e consumo & Não especificado \\
\hline
\caption{Temas para Estruturação e Acompanhamento}
\label{tab:temas_aplicacao}
\end{longtable}

\section{Formulário avaliativo  aplicação gamificada para auxilio dos professores do ensino fundamental I para o ensino de matemática básica}


O formulário foi estruturado para captar informações essenciais que permitirão o aprimoramento de recursos gamificados voltados especificamente para o ensino de matemática no ensino fundamental I, considerando as particularidades e necessidades deste nível educacional ele pode ser acessado atráves do link: \url{https://forms.gle/cqxpfYbaFTwmBVgSA}.


O objetivo é coletar feedback sobre a aplicação gamificada Math.Pow, que tem como finalidade auxiliar professores no ensino de matemática básica para alunos do Ensino Fundamental I. O formulário de avaliação é composto por várias seções que buscam abordar diferentes aspectos do aplicativo, desde a usabilidade e o design visual até a efetividade das ferramentas disponíveis e o engajamento dos alunos. Os dados coletados serão utilizados para ajustar o aplicativo de acordo com as necessidades reais, melhorar continuamente a experiência do usuário e validar as hipóteses iniciais sobre a efetividade do aplicativo na promoção do ensino de matemática básica. Para facilitar a navegabilidade, foi desenvolvido um site concentrando todo o conteúdo e a aplicação visite o site através do link: \url{https://mathpow.vercel.app/}.

O questionário elaborado tem como objetivo validar, junto a professores do Ensino Fundamental I, a eficácia e usabilidade de um aplicativo educacional que utiliza elementos de gamificação para o ensino de matemática. Para contextualizar os participantes, foi disponibilizado um vídeo demonstrativo do protótipo do aplicativo.

\subsection*{Principais Seções do Questionário}

\begin{enumerate}
    \item \textbf{Análise das Perguntas do Formulário:}
    \begin{itemize}
        \item \textbf{Eficácia da Gamificação:}
        \begin{itemize}
            \item Avalia se os professores acreditam que atividades gamificadas estimulam o interesse e engajamento dos alunos.
            \item Verifica a percepção sobre a eficácia das funcionalidades do aplicativo, como desafios e recompensas, no aprendizado de matemática.
        \end{itemize}
        \item \textbf{Funcionalidades do Aplicativo:}
        \begin{itemize}
            \item Identifica se os professores sentem falta de alguma funcionalidade crucial.
            \item Coleta sugestões de funcionalidades que poderiam ser adicionadas ao aplicativo.
        \end{itemize}
    \end{itemize}
    \item \textbf{Seções das Telas da Aplicação:}
    \begin{itemize}
        \item \textbf{Complemento ao Ensino:}
        \begin{itemize}
            \item Verifica se o aplicativo oferece recursos que complementam e contribuem com o ensino em sala de aula.
        \end{itemize}
        \item \textbf{Usabilidade e Design:}
        \begin{itemize}
            \item Avalia se a interface é fácil de navegar e se as instruções são claras e concisas.
            \item Analisa a adequação dos elementos visuais (cores, fontes, imagens) para o público-alvo.
            \item Confirma a presença de elementos de gamificação nas telas e a atratividade visual.
            \item Solicita sugestões para melhorar o design ou a organização das telas.
        \end{itemize}
    \end{itemize}
    \item \textbf{Conteúdo e Atividades:}
    \begin{itemize}
        \item \textbf{Adequação do Conteúdo:}
        \begin{itemize}
            \item Verifica se os assuntos abordados são condizentes com o público-alvo.
            \item Avalia se o plano de atividades proposto é válido para os alunos do Ensino Fundamental I.
        \end{itemize}
    \end{itemize}
    \item \textbf{Avaliação Geral:}
    \begin{itemize}
        \item \textbf{Intenção de Uso:}
        \begin{itemize}
            \item Identifica se os professores considerariam utilizar o aplicativo em atividades educacionais.
            \item Avalia a facilidade de implementação do aplicativo na rotina de ensino.
        \end{itemize}
        \item \textbf{Impressão Geral e Feedback:}
        \begin{itemize}
            \item Coleta a impressão geral dos professores sobre o aplicativo.
            \item Solicita observações, sugestões adicionais e identificação de pontos fortes e fracos.
        \end{itemize}
        \item \textbf{Avaliação Quantitativa:}
        \begin{itemize}
            \item Pede aos professores que atribuam uma nota ao aplicativo em uma escala de 0 a 10.
        \end{itemize}
    \end{itemize}
\end{enumerate}

O questionário foi estruturado para obter insights valiosos sobre a usabilidade, eficácia pedagógica e potencial de integração do aplicativo nas práticas educacionais. As respostas dos professores fornecerão orientações importantes para aprimorar o design, funcionalidades e conteúdo do aplicativo, garantindo que ele atenda às necessidades e expectativas do público-alvo.

\section{Formulário Avaliativo para Pais sobre o Aplicativo Gamificado de Matemática}

Este questionário foi elaborado para entender melhor as necessidades e experiências dos pais e das crianças em relação ao ensino da matemática por meio de ferramentas gamificadas. O formulário utilizado pode ser acessado através do link: \url{https://forms.gle/pCpiagsvjuW2HDMy5}

O objetivo deste formulário é coletar feedback dos pais sobre a aplicação gamificada \textit{Math.Pow}, que visa auxiliar no ensino de matemática básica para crianças do Ensino Fundamental I. O formulário é composto por várias seções que abordam diferentes aspectos do aplicativo, desde a usabilidade e design visual até a efetividade das atividades propostas e o engajamento das crianças. Os dados coletados serão utilizados para ajustar o aplicativo de acordo com as necessidades reais, melhorar continuamente a experiência do usuário e validar as hipóteses iniciais sobre a efetividade do aplicativo na promoção do ensino de matemática básica. Para facilitar o acesso, foi desenvolvido um site concentrando todo o conteúdo e a aplicação, que pode ser acessado em \url{https://mathpow.vercel.app/}.

O objetivo do \textbf{Formulário Avaliativo para Pais sobre o Aplicativo Gamificado de Matemática} é coletar feedback dos pais sobre a aplicação \textit{Math.Pow}, que auxilia no ensino de matemática básica para crianças do Ensino Fundamental I. O formulário abrange várias seções que avaliam aspectos como usabilidade, design visual, efetividade das atividades propostas e engajamento das crianças. Os dados coletados serão utilizados para ajustar o aplicativo conforme as necessidades reais, aprimorar a experiência do usuário e validar hipóteses sobre a efetividade do aplicativo no ensino de matemática básica. O acesso ao conteúdo e à aplicação é facilitado pelo site pode ser acessado através do link: \url{https://mathpow.vercel.app/}.

\subsection*{Estrutura do Formulário}

\begin{enumerate}
    \item \textbf{Apresentação (Seção 4 de 9)}
    \begin{itemize}
        \item Coleta informações básicas sobre o participante e seu filho, como nomes e idade.
    \end{itemize}
    \item \textbf{Conhecimento da Criança em Relação à Matemática (Seção 5 de 9)}
    \begin{itemize}
        \item Avalia o nível de conhecimento da criança em matemática.
        \item Inclui questões sobre familiaridade com números e operações básicas (soma, subtração, multiplicação e divisão).
    \end{itemize}
    \item \textbf{Desafios do Ensino (Seção 6 de 9)}
    \begin{itemize}
        \item Identifica os principais desafios enfrentados pelos pais no ensino de matemática aos filhos.
        \item Investiga o uso prévio de estratégias de gamificação.
    \end{itemize}
    \item \textbf{Satisfação (Seção 7 de 9)}
    \begin{itemize}
        \item Avalia o grau de satisfação com o aplicativo.
        \item Aborda aspectos como motivação da criança pelo sistema de pontuação e recompensas, e adequação do design das interfaces.
    \end{itemize}
    \item \textbf{Aceitação (Seção 8 de 9)}
    \begin{itemize}
        \item Investiga a aceitação e usabilidade do aplicativo.
        \item Verifica se a criança conseguiu utilizar o aplicativo com autonomia e interesse.
        \item Avalia a relevância e clareza do conteúdo matemático apresentado.
    \end{itemize}
    \item \textbf{Avaliação Geral (Seção 9 de 9)}
    \begin{itemize}
        \item Obtém uma avaliação global do aplicativo.
        \item Explora o potencial do aplicativo em melhorar a autoestima e confiança das crianças em relação à matemática.
        \item Coleta sugestões para aprimoramento do aplicativo.
    \end{itemize}
\end{enumerate}

O formulário foi estruturado para obter insights valiosos sobre a efetividade do aplicativo, permitindo ajustes e melhorias que atendam às necessidades dos usuários.

\subsection{Análise das Perguntas do Formulário}

Este questionário foi elaborado para entender melhor as necessidades e experiências dos pais e das crianças em relação ao ensino da matemática por meio de ferramentas gamificadas. As perguntas foram estruturadas para abranger:

\begin{itemize}
    \item \textbf{Perfil da Criança}: Coletar informações sobre a idade e o nível de conhecimento matemático da criança.
    \item \textbf{Desafios Enfrentados}: Identificar os principais obstáculos no ensino da matemática em casa.
    \item \textbf{Experiência com Gamificação}: Avaliar o uso prévio e a receptividade à gamificação como estratégia de ensino.
    \item \textbf{Satisfação com o Aplicativo}: Medir o grau de satisfação em relação às funcionalidades e design do aplicativo.
    \item \textbf{Aceitação e Usabilidade}: Verificar se o aplicativo é intuitivo e se promove autonomia na aprendizagem.
    \item \textbf{Benefícios Percebidos}: Coletar insights sobre como o aplicativo pode contribuir para o desenvolvimento da criança.
    \item \textbf{Sugestões de Melhoria}: Identificar pontos de melhoria e funcionalidades desejadas pelos usuários.
\end{itemize}

A análise das respostas permitirá ajustar o aplicativo \textit{Math.Pow} para melhor atender às necessidades dos usuários, aprimorando sua eficácia como ferramenta educacional para crianças do Ensino Fundamental I.
A aplicação do formulário é fundamental para compreender o impacto real do \textit{Math.Pow} no processo de aprendizagem das crianças e na experiência dos pais ao apoiar esse processo. O feedback coletado servirá como base para melhorias contínuas, visando oferecer uma solução educacional que seja não apenas eficaz no ensino da matemática básica, mas também envolvente e motivadora para as crianças.

\section{Análise dos Resultados do Formulário com Pais}

A pesquisa contou com a participação de 10 pais de crianças entre 6 e 10 anos de idade, fornecendo uma amostra diversificada para análise.

\subsection{Perfil das Crianças}

\begin{itemize}
    \item \textbf{Distribuição Etária:}
    \begin{itemize}
        \item 20\% das crianças têm 6 anos
        \item 20\% têm 7 anos
        \item 20\% têm 8 anos
        \item 20\% têm 9 anos
        \item 20\% têm 10 anos
    \end{itemize}
    
    \item \textbf{Conhecimento Matemático:}
    \begin{itemize}
        \item 80\% das crianças conhecem a representação dos números
        \item 60\% conseguem realizar operações de soma
        \item 50\% dominam a subtração
        \item 40\% realizam multiplicação
        \item 20\% conseguem fazer divisão
    \end{itemize}
\end{itemize}

\subsection{Principais Desafios Identificados}

\begin{itemize}
    \item \textbf{Desafios no Ensino:}
    \begin{itemize}
        \item Foco e concentração
        \item Dificuldade de aprendizagem
        \item Falta de interesse
        \item Falta de tempo dos pais
        \item Dificuldade com deveres de casa
    \end{itemize}
\end{itemize}

\subsection{Avaliação do Aplicativo}

\begin{itemize}
    \item \textbf{Aspectos Positivos:}
    \begin{itemize}
        \item 60\% dos pais relataram alta satisfação com o sistema de pontuação e recompensas
        \item 70\% consideraram o design das interfaces adequado
        \item 60\% concordaram que as atividades são apropriadas para o nível de desenvolvimento
    \end{itemize}
    
    \item \textbf{Aspectos a Melhorar:}
    \begin{itemize}
        \item 30\% dos pais indicaram necessidade de melhor suporte para iniciantes
        \item 20\% sugeriram mais elementos interativos
        \item 20\% demonstraram preferência por métodos tradicionais
    \end{itemize}
\end{itemize}

\subsection{Correlações Significativas}

\begin{itemize}
    \item \textbf{Idade e Desempenho:}
    \begin{itemize}
        \item Crianças de 9-10 anos apresentaram maior autonomia no uso do aplicativo
        \item Crianças de 6-7 anos necessitaram mais suporte parental
    \end{itemize}
    
    \item \textbf{Experiência Prévia:}
    \begin{itemize}
        \item 60\% dos pais já haviam utilizado estratégias de gamificação anteriormente
        \item Pais com experiência prévia em gamificação mostraram maior aceitação do aplicativo
    \end{itemize}
\end{itemize}

\subsection{Sugestões dos Pais}

\begin{itemize}
    \item Inclusão de personagens mais interativos
    \item Maior suporte para iniciantes
    \item Adaptação do conteúdo por nível de dificuldade
    \item Resolução de problemas de forma mais interativa
    \item Aprendizado através de jogos divertidos
\end{itemize}

\subsection{Conclusões Preliminares}

\begin{itemize}
    \item 70\% dos pais acreditam que o aplicativo pode beneficiar o aprendizado
    \item A aceitação é maior entre pais de crianças com melhor desempenho matemático prévio
    \item Existe uma correlação positiva entre a idade da criança e a autonomia no uso do aplicativo
    \item A gamificação é vista como uma ferramenta complementar eficaz por 60\% dos participantes
\end{itemize}

\subsection{Análise Detalhada dos Resultados da Pesquisa com Professores}

\subsubsection{Análise dos Dados}

A pesquisa contou com a participação de treze professores com idades entre 25 e 45 anos. Este aumento na amostra proporciona uma visão mais abrangente das percepções dos educadores em relação ao aplicativo.

\paragraph{Principais Padrões e Tendências}

\begin{itemize}
    \item \textbf{Percepção Positiva sobre Gamificação:}
    \begin{itemize}
        \item \textit{Melhoria na Autoestima e Confiança:} Dez dos treze professores acreditam que o uso de um aplicativo gamificado pode ajudar a melhorar a autoestima e a confiança das crianças em relação à matemática. Dois professores responderam "Talvez" e um respondeu "Não", indicando uma percepção majoritariamente positiva.
        \item \textit{Engajamento dos Alunos:} A maioria dos professores concorda que atividades gamificadas estimulam o interesse e o engajamento dos alunos.
    \end{itemize}
    \item \textbf{Eficácia das Funcionalidades:}
    \begin{itemize}
        \item Nove professores afirmaram que as funcionalidades do aplicativo (desafios, recompensas) são eficazes para o aprendizado de matemática.
        \item Alguns professores sugeriram melhorias, como a adição de mais jogos, suporte a múltiplos idiomas e avaliações de desempenho.
    \end{itemize}
    \item \textbf{Recursos Complementares:}
    \begin{itemize}
        \item A maioria dos participantes concorda que o aplicativo oferece recursos que complementam e contribuem com o ensino em sala de aula.
    \end{itemize}
    \item \textbf{Facilidade de Navegação e Design:}
    \begin{itemize}
        \item Muitos consideraram a interface fácil de navegar, com instruções claras e elementos visuais adequados.
        \item Houve sugestões para tornar a interface mais lúdica e melhorar o design para atrair mais o público infantil.
    \end{itemize}
\end{itemize}

\paragraph{Correlações Relevantes}

\begin{itemize}
    \item \textbf{Idade e Aceitação Tecnológica:}
    \begin{itemize}
        \item Professores na faixa etária acima de 37 anos mostraram-se mais céticos quanto à eficácia direta do aplicativo, indicando uma possível correlação entre idade e aceitação de novas tecnologias educacionais.
    \end{itemize}
    \item \textbf{Experiência Profissional e Implementação:}
    \begin{itemize}
        \item Alguns professores reconheceram desafios na implementação do aplicativo, especialmente relacionados ao acesso à internet e à realidade das escolas públicas.
    \end{itemize}
\end{itemize}

\paragraph{Pontos de Destaque e Discrepâncias}

\begin{itemize}
    \item \textbf{Acesso nas Escolas Públicas:}
    \begin{itemize}
        \item Preocupações foram levantadas sobre a disponibilidade do aplicativo em escolas públicas, onde o ensino é menos lúdico e há limitações de recursos tecnológicos.
    \end{itemize}
    \item \textbf{Dependência de Internet:}
    \begin{itemize}
        \item A necessidade de conexão com a internet foi apontada como um possível obstáculo, sugerindo a importância de considerar funcionalidades offline.
    \end{itemize}
\end{itemize}

\subsubsection{Insights por Categoria}

\paragraph{Categoria: Opinião sobre Eficácia da Gamificação}

\begin{itemize}
    \item \textbf{Positivos:}
    \begin{itemize}
        \item Professores que acreditam na eficácia destacam o potencial do aplicativo em auxiliar de forma divertida, aumentando o interesse pela matéria.
        \item Reconhecem que o lúdico ajuda na motivação e, consequentemente, no aprendizado.
    \end{itemize}
    \item \textbf{Neutros/Céticos:}
    \begin{itemize}
        \item Alguns professores expressaram dúvida sobre a eficácia direta, mas reconheceram o valor como ferramenta de suporte.
    \end{itemize}
    \item \textbf{Negativos:}
    \begin{itemize}
        \item Um professor não acredita que o aplicativo seja eficaz, apontando que não atende às necessidades dos alunos e destacando aspectos como design pouco atrativo.
    \end{itemize}
\end{itemize}

\paragraph{Categoria: Facilidades e Dificuldades na Implementação}

\begin{itemize}
    \item \textbf{Facilidades:}
    \begin{itemize}
        \item Muitos professores consideram o aplicativo fácil de implementar em suas rotinas de ensino, apontando a integração como "mais uma ferramenta para a aprendizagem".
    \end{itemize}
    \item \textbf{Dificuldades:}
    \begin{itemize}
        \item Preocupações quanto à acessibilidade em escolas públicas e a dependência de internet podem dificultar a adoção plena do aplicativo.
        \item Alguns mencionaram que o processo de instalação foi complicado.
    \end{itemize}
\end{itemize}

\subsubsection{Visualização dos Resultados}

\paragraph{Sugestões de Representações Visuais}

\begin{itemize}
    \item \textbf{Gráfico de Barras:}
    \begin{itemize}
        \item \textit{Eficácia Percebida das Funcionalidades:} Comparar o número de professores que consideram as funcionalidades eficazes versus aqueles que têm dúvidas ou não as consideram eficazes.
        \item \textit{Intenção de Uso:} Mostrar quantos professores considerariam utilizar o aplicativo em atividades educacionais.
        \item \textit{Notas Atribuídas ao Aplicativo:} Exibir a distribuição das notas dadas pelos professores, que variaram de 2 a 10.
    \end{itemize}
    \item \textbf{Tabela Resumo:}
    \begin{itemize}
        \item \textit{Feedback e Sugestões:} Listar as sugestões e observações feitas pelos professores para fácil referência.
    \end{itemize}
    \item \textbf{Gráfico de Pizza:}
    \begin{itemize}
        \item \textit{Percepção sobre Estímulo ao Engajamento:} Percentual de professores que acreditam que atividades gamificadas estimulam o interesse dos alunos.
    \end{itemize}
\end{itemize}

\paragraph{Aspectos a Serem Enfatizados Visualmente}

\begin{itemize}
    \item A predominância de opiniões positivas sobre a gamificação e seu impacto no engajamento dos alunos.
    \item As preocupações com a implementação prática, especialmente em relação ao acesso à internet e recursos necessários nas escolas.
    \item A gama de sugestões para melhoria, indicando o envolvimento dos professores e o potencial para evoluir o aplicativo.
\end{itemize}

\subsubsection{Conclusões e Recomendações}

\paragraph{Principais Achados}

\begin{itemize}
    \item Há uma percepção geral positiva sobre o potencial do aplicativo em auxiliar no ensino de matemática de forma lúdica e engajadora.
    \item Professores reconhecem os benefícios da gamificação para aumentar o interesse e a motivação dos alunos.
    \item Existem preocupações reais sobre a acessibilidade do aplicativo em diferentes contextos educacionais, especialmente em escolas públicas com recursos limitados.
\end{itemize}

\paragraph{Insights Acionáveis}

\begin{itemize}
    \item \textbf{Desenvolvimento de Funcionalidades Offline:}
    \begin{itemize}
        \item Considerar a implementação de recursos que permitam o uso do aplicativo sem a necessidade constante de internet.
    \end{itemize}
    \item \textbf{Aprimoramento do Design:}
    \begin{itemize}
        \item Tornar a interface mais lúdica e atrativa, incorporando elementos visuais que sejam mais envolventes para o público infantil.
    \end{itemize}
    \item \textbf{Inclusão e Acesso:}
    \begin{itemize}
        \item Desenvolver estratégias para ampliar o acesso em escolas públicas, possivelmente por meio de parcerias ou versões adaptadas.
    \end{itemize}
\end{itemize}

\paragraph{Próximos Passos para Aprofundar a Análise}

\begin{itemize}
    \item \textbf{Ampliação da Amostra:}
    \begin{itemize}
        \item Realizar a pesquisa com um número maior de professores para obter dados mais representativos.
    \end{itemize}
    \item \textbf{Diversificação dos Participantes:}
    \begin{itemize}
        \item Incluir professores de diferentes regiões e contextos socioeconômicos para entender melhor as variações nas percepções e necessidades.
    \end{itemize}
    \item \textbf{Teste Piloto em Escolas:}
    \begin{itemize}
        \item Implementar o aplicativo em ambiente escolar controlado para observar a interação real dos alunos e coletar feedback direto.
    \end{itemize}
\end{itemize}







