\chapter{Experimento e Coleta de Dados} \label{chp:exp}

\section{Análise da Matriz Curricular do Ensino Fundamental 1 (4º Ano) na Matéria de Matemática}

A análise da matriz curricular do Ensino Fundamental 1, especificamente para o 4º ano na matéria de Matemática, envolve a comparação entre o conteúdo programático do Colégio Salesiano do Salvador e o Referencial Curricular Municipal de Salvador. Ambos os documentos têm como objetivo proporcionar uma educação matemática abrangente, mas diferem em sua abordagem e detalhamento.

\subsection{Conteúdo Programático do Colégio Salesiano do Salvador}

No Colégio Salesiano do Salvador, o currículo de Matemática para o 4º ano é dividido em três trimestres, cada um abordando tópicos específicos. No primeiro trimestre, os alunos são introduzidos aos números romanos e ao sistema de numeração decimal na base 10, com foco em números até a dezena de milhar. Além disso, há uma ênfase na composição e decomposição de números, valor posicional, arredondamento e estimativa.

Durante o segundo trimestre, o currículo avança para multiplicação e divisão, introduzindo conceitos como múltiplos, propriedades da multiplicação e divisão por 10, 100 e 1000. A geometria é também abordada, com os alunos aprendendo sobre perímetro e a ideia de área. Estatística e probabilidade são introduzidas, com foco em variáveis categóricas e numéricas e na criação e interpretação de tabelas e gráficos.

No terceiro trimestre, o currículo foca em números decimais, abordando sua utilização no sistema monetário brasileiro, bem como adição e subtração de números decimais. Geometria continua a ser explorada, incluindo conceitos de ângulos (retos, agudos e obtusos), retas paralelas e perpendiculares, simetria, prismas e pirâmides. A educação financeira é introduzida, ensinando conceitos de consumo, valores de vida e empreendedorismo \cite{conteudo_programatico_salesiano}.

\subsection{Referencial Curricular Municipal de Salvador}

O Referencial Curricular Municipal para os anos iniciais do ensino fundamental de Salvador apresenta uma abordagem integrada e contínua para a aprendizagem matemática. Este currículo também abrange números e operações, geometria, grandezas e medidas, e tratamento da informação, mas com uma estrutura menos fragmentada por trimestre.

Os alunos são incentivados a resolver problemas que envolvem adição, subtração, multiplicação e divisão com números naturais, além de realizar diferentes tipos de cálculos (exatos, aproximados e mentais). A geometria é integrada ao aprendizado diário, com foco na identificação, descrição e representação da localização e movimentação de objetos no espaço, bem como na análise e comparação de figuras geométricas planas e espaciais.

Grandezas e medidas são abordadas através da resolução de problemas que envolvem a estimativa e medição de diferentes quantidades, como tempo, comprimento, massa e capacidade. Os alunos aprendem a utilizar medidas de tempo e a realizar conversões simples, além de aplicar esses conceitos em situações práticas.

O tratamento da informação é enfatizado com a coleta e interpretação de dados, apresentando resultados em tabelas e gráficos de colunas. Os alunos são ensinados a coletar dados de duas variáveis, a criar tabelas e gráficos simples e de dupla entrada, e a interpretar as informações apresentadas \cite{referencial_curricular_municipal}.

\section{Comparação entre Matrizes de Ensino: Temas e Abordagens Principais}

Comparando os dois currículos, foi observado que ambos cobrem os principais tópicos de Matemática necessários para o 4º ano, mas com diferenças notáveis em sua abordagem. O currículo do Colégio Salesiano do Salvador é mais detalhado e dividido claramente por trimestres, oferecendo uma estrutura bem definida para cada período letivo. Em contraste, o Referencial Curricular Municipal de Salvador adota uma abordagem mais integrada, com foco na aplicação prática dos conhecimentos e na resolução de problemas contínua ao longo do ano.

A inclusão de educação financeira no currículo do Colégio Salesiano é um diferencial, pois aborda aspectos como consumo, valores de vida e empreendedorismo, preparando os alunos para lidar com questões financeiras desde cedo. Por outro lado, o Referencial Curricular Municipal de Salvador enfatiza a interpretação de dados e a aplicação prática dos conceitos matemáticos, promovendo uma compreensão mais ampla e contextualizada da matemática no dia a dia dos alunos.

Ambos os currículos são eficazes em seus objetivos, oferecendo uma educação matemática completa e adequada para os alunos do 4º ano, mas com metodologias e enfoques distintos que refletem suas respectivas filosofias educacionais.

\begin{longtable}{|p{3cm}|p{6cm}|p{6cm}|}
 
\hline
\textbf{Tópicos} & \textbf{Colégio Salesiano do Salvador} & \textbf{Referencial Curricular Municipal de Salvador} \\
\hline
Números e Operações & Números romanos, sistema de numeração decimal, números até dezena de milhar, adição, subtração, multiplicação, divisão & Relações subjacentes a números, operações com números naturais, cálculos exatos, aproximados e mentais \\
\hline
Geometria & Perímetro, área, ângulos, retas paralelas e perpendiculares, prismas, pirâmides, simetria & Localização e movimentação de objetos, figuras geométricas planas e espaciais, ângulos retos e não retos \\
\hline
Estatística e Probabilidade & Variáveis categóricas e numéricas, tabelas, gráficos, gráfico pictórico & Estimativa e medição de tempo, comprimento, massa, capacidade, conversões simples \\
\hline
Grandezas e Medidas & Medidas de comprimento, massa, capacidade, tempo, temperatura & Estimativa de diferentes medidas, medições efetivas, uso de medidas de tempo \\
\hline
Educação Financeira & Consumo, valores de vida, crenças, paradigmas, aposentadoria, proteção, empreendedorismo, metas, cooperativismo, renda e consumo & Não especificado \\
\hline
Tratamento da Informação & Coleta de dados, apresentação em tabelas e gráficos & Coleta e interpretação de dados, criação de tabelas e gráficos simples e de dupla entrada \\
\hline
\caption{Tabela comparativa entre matrizes curriculares de matemática básica do ensino fundamental 1, 4º ano}
\label{tab:comparacao_matrizes}
\end{longtable}

\section{Estruturação e Acompanhamento dos Temas Abordados na Aplicação}

Para enriquecer essa análise, utilizou-se o Khan Academy como uma base de dados para coleta de assuntos, garantindo uma abordagem ampla e atualizada dos tópicos. Baseados no conteúdo programático anterior, chegamos ao filtro dos conteúdos mais relevantes para a aplicação. A plataforma fornece uma vasta gama de materiais educativos, exercícios práticos e vídeos explicativos que cobrem os tópicos abordados nas matrizes curriculares \cite{khanacademy_numeros_soma_subtracao}.

\begin{longtable}{|p{2cm}|p{4cm}|p{4cm}|p{4cm}|}
\hline
\textbf{Unidade} & \textbf{Tópicos no Khan Academy} & \textbf{Colégio Salesiano do Salvador} & \textbf{Referencial Curricular Municipal do Salvador} \\
\hline
Unidade 1 & Números: soma e subtração & Adição, subtração, propriedades da adição, expressões numéricas, situações-problema & Relações subjacentes a números, operações com números naturais, cálculos exatos, aproximados e mentais \\
\hline
Unidade 2 & Números: multiplicação e divisão & Multiplicação, divisão, propriedades da multiplicação, regularidades da divisão, situações-problema & Relações subjacentes a números, operações com números naturais, cálculos exatos, aproximados e mentais \\
\hline
Unidade 3 & Números: frações & Não especificado & Noção de fração, leitura de fração, comparação de frações, adição e subtração de frações, situações-problema \\
\hline
Unidade 4 & Álgebra & Não especificado & Noção de equação, valor desconhecido, relações de igualdade, operações inversas \\
\hline
Unidade 5 & Geometria & Perímetro, área, ângulos, retas paralelas e perpendiculares, prismas, pirâmides, simetria & Localização e movimentação de objetos, figuras geométricas planas e espaciais, ângulos retos e não retos \\
\hline
Unidade 6 & Grandezas e medidas & Medidas de comprimento, massa, capacidade, tempo, temperatura & Estimativa de diferentes medidas, medições efetivas, uso de medidas de tempo \\
\hline
Unidade 7 & Probabilidade e estatística & Variáveis categóricas e numéricas, tabelas, gráficos, gráfico pictórico & Coleta e interpretação de dados, criação de tabelas e gráficos simples e de dupla entrada \\
\hline
Unidade 8 & Educação financeira & Consumo, valores de vida, crenças, paradigmas, aposentadoria, proteção, empreendedorismo, metas, cooperativismo, renda e consumo & Não especificado \\
\hline
\caption{Temas para Estruturação e Acompanhamento}
\label{tab:temas_aplicacao}
\end{longtable}

\section{Formulário avaliativo  aplicação gamificada para auxilio dos professores do ensino fundamental I para o ensino de matemática básica}

O objetivo é coletar feedback sobre a aplicação gamificada Math.Pow, que tem como finalidade auxiliar professores no ensino de matemática básica para alunos do Ensino Fundamental I. O formulário de avaliação é composto por várias seções que buscam abordar diferentes aspectos do aplicativo, desde a usabilidade e o design visual até a efetividade das ferramentas disponíveis e o engajamento dos alunos. Os dados coletados serão utilizados para ajustar o aplicativo de acordo com as necessidades reais, melhorar continuamente a experiência do usuário e validar as hipóteses iniciais sobre a efetividade do aplicativo na promoção do ensino de matemática básica. Para facilitar a navegabilidade, foi desenvolvido um site concentrando todo o conteúdo e a aplicação visite o site clicando no nome \href{https://mathpow.vercel.app/}{MathPow}.

O questionário elaborado tem como objetivo validar, junto a professores do Ensino Fundamental I, a eficácia e usabilidade de um aplicativo educacional que utiliza elementos de gamificação para o ensino de matemática. Para contextualizar os participantes, foi disponibilizado um vídeo demonstrativo do protótipo do aplicativo.

\subsection*{Principais Seções do Questionário}

\begin{enumerate}
    \item \textbf{Análise das Perguntas do Formulário:}
    \begin{itemize}
        \item \textbf{Eficácia da Gamificação:}
        \begin{itemize}
            \item Avalia se os professores acreditam que atividades gamificadas estimulam o interesse e engajamento dos alunos.
            \item Verifica a percepção sobre a eficácia das funcionalidades do aplicativo, como desafios e recompensas, no aprendizado de matemática.
        \end{itemize}
        \item \textbf{Funcionalidades do Aplicativo:}
        \begin{itemize}
            \item Identifica se os professores sentem falta de alguma funcionalidade crucial.
            \item Coleta sugestões de funcionalidades que poderiam ser adicionadas ao aplicativo.
        \end{itemize}
    \end{itemize}
    \item \textbf{Seções das Telas da Aplicação:}
    \begin{itemize}
        \item \textbf{Complemento ao Ensino:}
        \begin{itemize}
            \item Verifica se o aplicativo oferece recursos que complementam e contribuem com o ensino em sala de aula.
        \end{itemize}
        \item \textbf{Usabilidade e Design:}
        \begin{itemize}
            \item Avalia se a interface é fácil de navegar e se as instruções são claras e concisas.
            \item Analisa a adequação dos elementos visuais (cores, fontes, imagens) para o público-alvo.
            \item Confirma a presença de elementos de gamificação nas telas e a atratividade visual.
            \item Solicita sugestões para melhorar o design ou a organização das telas.
        \end{itemize}
    \end{itemize}
    \item \textbf{Conteúdo e Atividades:}
    \begin{itemize}
        \item \textbf{Adequação do Conteúdo:}
        \begin{itemize}
            \item Verifica se os assuntos abordados são condizentes com o público-alvo.
            \item Avalia se o plano de atividades proposto é válido para os alunos do Ensino Fundamental I.
        \end{itemize}
    \end{itemize}
    \item \textbf{Avaliação Geral:}
    \begin{itemize}
        \item \textbf{Intenção de Uso:}
        \begin{itemize}
            \item Identifica se os professores considerariam utilizar o aplicativo em atividades educacionais.
            \item Avalia a facilidade de implementação do aplicativo na rotina de ensino.
        \end{itemize}
        \item \textbf{Impressão Geral e Feedback:}
        \begin{itemize}
            \item Coleta a impressão geral dos professores sobre o aplicativo.
            \item Solicita observações, sugestões adicionais e identificação de pontos fortes e fracos.
        \end{itemize}
        \item \textbf{Avaliação Quantitativa:}
        \begin{itemize}
            \item Pede aos professores que atribuam uma nota ao aplicativo em uma escala de 0 a 10.
        \end{itemize}
    \end{itemize}
\end{enumerate}

O questionário foi estruturado para obter insights valiosos sobre a usabilidade, eficácia pedagógica e potencial de integração do aplicativo nas práticas educacionais. As respostas dos professores fornecerão orientações importantes para aprimorar o design, funcionalidades e conteúdo do aplicativo, garantindo que ele atenda às necessidades e expectativas do público-alvo.

\section{Formulário Avaliativo para Pais sobre o Aplicativo Gamificado de Matemática}

O objetivo deste formulário é coletar feedback dos pais sobre a aplicação gamificada \textit{Math.Pow}, que visa auxiliar no ensino de matemática básica para crianças do Ensino Fundamental I. O formulário é composto por várias seções que abordam diferentes aspectos do aplicativo, desde a usabilidade e design visual até a efetividade das atividades propostas e o engajamento das crianças. Os dados coletados serão utilizados para ajustar o aplicativo de acordo com as necessidades reais, melhorar continuamente a experiência do usuário e validar as hipóteses iniciais sobre a efetividade do aplicativo na promoção do ensino de matemática básica. Para facilitar o acesso, foi desenvolvido um site concentrando todo o conteúdo e a aplicação, que pode ser acessado em \href{https://mathpow.vercel.app/}{MathPow}.

O objetivo do \textbf{Formulário Avaliativo para Pais sobre o Aplicativo Gamificado de Matemática} é coletar feedback dos pais sobre a aplicação \textit{Math.Pow}, que auxilia no ensino de matemática básica para crianças do Ensino Fundamental I. O formulário abrange várias seções que avaliam aspectos como usabilidade, design visual, efetividade das atividades propostas e engajamento das crianças. Os dados coletados serão utilizados para ajustar o aplicativo conforme as necessidades reais, aprimorar a experiência do usuário e validar hipóteses sobre a efetividade do aplicativo no ensino de matemática básica. O acesso ao conteúdo e à aplicação é facilitado pelo site \href{https://mathpow.vercel.app/}{MathPow}.

\subsection*{Estrutura do Formulário}

\begin{enumerate}
    \item \textbf{Apresentação (Seção 4 de 9)}
    \begin{itemize}
        \item Coleta informações básicas sobre o participante e seu filho, como nomes e idade.
    \end{itemize}
    \item \textbf{Conhecimento da Criança em Relação à Matemática (Seção 5 de 9)}
    \begin{itemize}
        \item Avalia o nível de conhecimento da criança em matemática.
        \item Inclui questões sobre familiaridade com números e operações básicas (soma, subtração, multiplicação e divisão).
    \end{itemize}
    \item \textbf{Desafios do Ensino (Seção 6 de 9)}
    \begin{itemize}
        \item Identifica os principais desafios enfrentados pelos pais no ensino de matemática aos filhos.
        \item Investiga o uso prévio de estratégias de gamificação.
    \end{itemize}
    \item \textbf{Satisfação (Seção 7 de 9)}
    \begin{itemize}
        \item Avalia o grau de satisfação com o aplicativo.
        \item Aborda aspectos como motivação da criança pelo sistema de pontuação e recompensas, e adequação do design das interfaces.
    \end{itemize}
    \item \textbf{Aceitação (Seção 8 de 9)}
    \begin{itemize}
        \item Investiga a aceitação e usabilidade do aplicativo.
        \item Verifica se a criança conseguiu utilizar o aplicativo com autonomia e interesse.
        \item Avalia a relevância e clareza do conteúdo matemático apresentado.
    \end{itemize}
    \item \textbf{Avaliação Geral (Seção 9 de 9)}
    \begin{itemize}
        \item Obtém uma avaliação global do aplicativo.
        \item Explora o potencial do aplicativo em melhorar a autoestima e confiança das crianças em relação à matemática.
        \item Coleta sugestões para aprimoramento do aplicativo.
    \end{itemize}
\end{enumerate}

O formulário foi estruturado para obter insights valiosos sobre a efetividade do aplicativo, permitindo ajustes e melhorias que atendam às necessidades dos usuários.

\subsection{Análise das Perguntas do Formulário}

Este questionário foi elaborado para entender melhor as necessidades e experiências dos pais e das crianças em relação ao ensino da matemática por meio de ferramentas gamificadas. As perguntas foram estruturadas para abranger:

\begin{itemize}
    \item \textbf{Perfil da Criança}: Coletar informações sobre a idade e o nível de conhecimento matemático da criança.
    \item \textbf{Desafios Enfrentados}: Identificar os principais obstáculos no ensino da matemática em casa.
    \item \textbf{Experiência com Gamificação}: Avaliar o uso prévio e a receptividade à gamificação como estratégia de ensino.
    \item \textbf{Satisfação com o Aplicativo}: Medir o grau de satisfação em relação às funcionalidades e design do aplicativo.
    \item \textbf{Aceitação e Usabilidade}: Verificar se o aplicativo é intuitivo e se promove autonomia na aprendizagem.
    \item \textbf{Benefícios Percebidos}: Coletar insights sobre como o aplicativo pode contribuir para o desenvolvimento da criança.
    \item \textbf{Sugestões de Melhoria}: Identificar pontos de melhoria e funcionalidades desejadas pelos usuários.
\end{itemize}

A análise das respostas permitirá ajustar o aplicativo \textit{Math.Pow} para melhor atender às necessidades dos usuários, aprimorando sua eficácia como ferramenta educacional para crianças do Ensino Fundamental I.

\subsection{Conclusão}

A aplicação do formulário é fundamental para compreender o impacto real do \textit{Math.Pow} no processo de aprendizagem das crianças e na experiência dos pais ao apoiar esse processo. O feedback coletado servirá como base para melhorias contínuas, visando oferecer uma solução educacional que seja não apenas eficaz no ensino da matemática básica, mas também envolvente e motivadora para as crianças.
