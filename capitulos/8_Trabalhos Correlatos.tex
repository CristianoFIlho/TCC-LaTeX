\chapter{Trabalhos Correlatos}\label{chp:conc}


\section{Uso de jogos digitais nas aulas como motivadores da aprendizagem}

Os jogos digitais têm se mostrado uma estratégia eficaz para motivar o aprendizado em sala de aula. Pesquisas indicam que eles podem aumentar o engajamento dos alunos, promovendo um aprendizado mais ativo e participativo. Os jogos criam um ambiente interativo e dinâmico, onde os alunos podem experimentar, falhar e tentar novamente sem medo de consequências negativas, incentivando a curiosidade e o desejo de aprender. A utilização de jogos digitais também permite a personalização do ensino, adaptando-se às necessidades e ao ritmo de cada aluno.

\cite{landers2017}


\section{Uma experiência no ensino fundamental, com gamificação no ensino fundamental I}

A gamificação tem sido usada com sucesso no ensino fundamental I para melhorar o envolvimento dos alunos e facilitar a compreensão de conceitos complexos. Uma experiência significativa ocorreu em uma escola pública, onde elementos de jogos foram incorporados ao currículo de matemática. Os alunos eram incentivados a acumular pontos, ganhar distintivos e subir de nível à medida que completavam tarefas e desafios. Essa abordagem não apenas aumentou a motivação e o interesse pelo assunto, mas também resultou em melhorias significativas no desempenho acadêmico, demonstrando a eficácia da gamificação no contexto educacional.

\cite{kapp2012gamification}

\section{Proposta de ferramenta de gamificação no aprendizado da matemática com alunos do ensino fundamental}

Para melhorar o aprendizado de matemática no ensino fundamental, sugere-se uma ferramenta de gamificação que combine princípios de design de jogos com objetivos educacionais. A ferramenta incluiria módulos interativos que abrangem tópicos como aritmética, álgebra e geometria. Cada módulo proporia desafios progressivos, recompensas instantâneas e feedback contínuo para manter os alunos engajados. Além disso, a ferramenta permitiria personalização de acordo com as necessidades individuais dos alunos, fornecendo suporte adicional ou desafios avançados conforme necessário. Essa abordagem tem como objetivo tornar o aprendizado de matemática mais divertido, eficaz e acessível.

\cite{tolomei2017gamificacao}


