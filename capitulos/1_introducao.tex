\chapter{Introdução}\label{chp:intro}

Em síntese, refletir de forma lúdica e interativa, pode estabelecer uma conexão com os alunos interligados à matemática. Se fez necessário a criação de uma ideologia de  aplicação representado por um personagem chamado Polvo Pi, desenvolvido para conduzir e motivar principalmente o público infantil e interessados no aprendizado através da aplicação educativa. A missão principal do persona é ajudar o colegiado do Ensino Fundamental I, trazendo vínculos e identificações por parte dos estudantes, despertando a descobrirem como a matemática pode ser divertida e fascinante, utilizando jogos, desafios e histórias envolventes para ensinar conceitos matemáticos de maneira simples, conduzindo a imersão em um novo mundo numa troca constante de conhecimento. \cite{Lopes2020}

Polvo Pi acredita que as dificuldades enfrentadas no ensino da matemática podem ser superadas com o auxílio da tecnologia como ferramenta, bastando encontrar a maneira certa de aprender. A aplicação tem como principal objetivo viabilizar a elaboração de metodologias de ensino através de planejamentos pedagógicos eficientes e criativos. A escolha do melhor nome para representar essa nova ferramenta, está coerente a fatores científicos, relações e comparações. Sendo assim o nome escolhido é “POLVO PI”:

POLVO: é um animal marinho de corpo mole e flexível, possuindo oito braços chamados tentáculos, conhecido por sua inteligência, habilidades de camuflagem e comportamentos complexos; ele serve como inspiração para a idealização da aplicação educativa. Da família dos cefalópodes, este molusco busca se adaptar a uma variedade de ambientes para sobreviver. Por analogia, sob o mesmo ponto de vista. Do mesmo modo, a ideia em forma de aplicação se adapta às diferentes maneiras de aprendizado dos alunos, utilizando as características-chave desse fascinante mamífero pelágico almejando tornar o ensino mais eficaz e envolvente.\cite{BBC2023}

PI: uma constante matemática que representa a relação entre a circunferência de um círculo e seu diâmetro. É uma proporção numérica infinita, ou seja, sua sequência de números sem fim; assim como o aprendizado de uma criança. Além disso, a percepção da complexidade na comparação da inteligência do homem relacionada ao polvo é excepcional. Ponderando que mais da metade dos 500 milhões de neurônios do animal se concentram em seus tentáculos. Isso significa que cada um deles pode agir sozinho ou em coordenação com os demais. E, enquanto o cérebro humano é visto como um controlador central, a inteligência do polvo pode estar distribuída em uma rede de neurônios, um pouco como a internet. \cite{BrasilEscola2023}



\section{Aplicabilidade e Motivação}\label{chp:aplic}

A dificuldade encontrada na docência do Ensino Fundamental I nas atribuições das aulas de matemática, tratam-se de abordagens que proporcionam alternâncias e intercorrências consideráveis. Os problemas, contratempos e adversidades apresentam-se desde os planejamentos pedagógicos por coordenadores em planos de cursos e sobrevêm chegando até os docentes em seus planos de aulas que não conseguem atingir com eficiência o entendimento e absorção da aprendizagem pelos discentes. 

Tanto o magistério quanto o alunato enfrentam objeções ao lidar com essa disciplina, procurando superar muitos obstáculos, ponderando que os incentivos deparam em grande parte dos estudantes que não consideram a matemática como algo notável ou concebível em suas vivências. Além disso, os estímulos para aprendizagem, aplicação das suas concepções e ideias são diretamente impactados. Igualmente a subjetividade e complexidão da matemática abrange concepções subjetivas e constantemente abstrusa, existindo a possibilidade do alunado encontrar obstruções para perceber e assimilar os conceitos, o que pode trazer o desapontamento. 

Da mesma forma a fixação e cognição dos estudantes focalizam em memorar os métodos e suas fórmulas em vez de tentar compreender os princípios implícitos subentendidos, impossibilitando a sua criatividade para solucionar determinadas questões com eficacia. Assim também as ferramentas necessárias para os educadores deparam-se com impedimentos para obter materiais instrutivos, compatíveis e metodologias de transmissão de conhecimentos eficientes, considerando a probabilidade de impactos negativos na natureza da orientação. Por analogia a aflição pela disciplina é perceptível em grande parte dos alunos que se apresentam ansiosos, pois o processo de aprender matemática oferece tendências que afetam a conduta e autoestima. 

Atualmente a maior parte dos jovens e até mesmo alguns adultos vivenciam a sensação de experienciar jogos com parâmetros físicos ou digitais por intermédio de videogames, computadores, smartphones, dominós, baralhos, dados etc. Subsiste neles alguma coisa indeterminada que atrai e magnetiza, tomando por completo a atenção do usuário. Que idealiza a sobrelevação, individualmente ou em equipe, incentivado com o resultado alcançado em suas fases, etapas, níveis e etc. Pois mesmo parecendo uma simples diversão, nas variadas formas e possibilidades de jogos se aplica energia espontaneamente. Além disso as atividades de entretenimentos, define a estruturação intrínseca mediante a diretrizes que tem por objetivo conduzir a atividade com procedimentos organizados.

\section{Contextualização}

A gamificação, que utiliza elementos de jogos para engajar e motivar os alunos, tem se destacado como uma abordagem eficaz para o ensino. A gamificação oferece oportunidades de aprendizado interativo, onde desafios, recompensas e feedbacks imediatos podem tornar a experiência educacional mais estimulante e atraente para as crianças.

A evolução das tecnologias tem facilitado significativamente a inclusão de pessoas com deficiência, proporcionando maior acessibilidade e autonomia. Novos recursos e APIs que facilitam a interconexão e integração entre plataformas, têm possibilitado o desenvolvimento de aplicações e soluções tecnológicas que atendam a um número cada vez maior de usuários com necessidades especiais. Um exemplo disso é o aplicativo "Hand Talk", que disponibiliza um plugin para integrações de acessibilidade a deficientes auditivos, permitindo tradução automática para Língua Brasileira de Sinais.

Outras iniciativas importantes são ferramentas de leitura de tela, ampliação de conteúdo, contraste elevado e interfaces adaptáveis. Essas inovações têm o poder de reduzir barreiras, promover inclusão digital e transformar positivamente a vida de pessoas com deficiência. Como destaca  \cite{furlan2016desenvolvimento}, "a tecnologia tem um papel crucial na promoção da acessibilidade e inclusão das pessoas com deficiência".


\section{Objetivos}

Afinal de contas, qual é a definição de gamificação? Quais são os pontos de uniformidade, divergência e convergência entre gamificação e jogos? Como pôr em prática? Neste estudo acadêmico, a abordagem da gamificação será abrangente, com reflexões e idealizações que se referem a um tema atual que estabelece relações de grande importância no âmbito da educação.O elemento conceitual investigativo deste objeto de pesquisa é muito significativo, dado que a gamificação requer um entendimento satisfatório para ser empregue com eficiência, sensatez e seriedade. A temática será abordada em tópicos considerados relevantes. Investigando as concepções e ideias empregadas, os recursos e componentes adotados, referenciando determinadas finalidades. Destarte, busca-se obter um entendimento referente ao conteúdo discutido neste trabalho de pesquisa e investigação científica, considerando que esta abordagem dissertada se trata de uma esfera que denota grande relevância mostrando considerável importância no campo acadêmico.


\subsection{Objetivo Geral}

 O principal objetivo desse trabalho de conclusão de curso é analisar o uso da gamificação através de um aplicativo,  enquanto uma metodologia ativa , para o ensino de matemática, com  alunos do fundamental I. Como ferramenta para auxílio na pesquisa temos como objetivo inicial a produção um aplicativo que usa elementos de gamificação, a escolha de um aplicativo é que não tenha a necessidade de ser usado somente em sala de aula, e sim para servir como ferramenta auxiliar no processo de ensino.


\subsection{Objetivos Específicos}

\begin{itemize}
\item{Autenticar se o uso de jogos educacionais para o ensino de matemática é eficaz.
}
\item {Avaliar quais conteúdos são mais estruturantes para o ensino da matemática no fundamental I.
} 
\item {Desenvolver sequências didáticas com a metodologia da gamificação.
 }  
\item {Desenvolver um aplicativo com uso das técnicas de gamificação.
}
\item {Validar o uso do aplicativo com professores e estudantes.
}
\item {Auxiliar no ensino de matemática para alunos do ensino fundamental I.
}
\end{itemize}

\section{Estrutura do Trabalho}

No Capítulo 1 é apresentado o objetivo geral e objetivos específicos, enquanto que o Capítulo 2 traz a Fundamentação Teórica. No Capítulo 3 está apresentada a metodologia do trabalho, enquanto que no Capítulo 4 são apresentados o experimento e coleta de dados. Finalmente, no Capítulo 5 é apresentada a conclusão desta monografia.


