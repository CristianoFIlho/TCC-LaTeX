\chapter{Introdução}\label{chp:intro}

O ensino no Brasil, assim como em diversos outros países, enfrenta desafios significativos na busca por metodologias que promovam a aprendizagem eficaz e o engajamento dos alunos. Entre as disciplinas que mais sofrem com esses desafios, a matemática se destaca devido à sua complexidade e à percepção negativa que muitos alunos têm em relação a ela.

A Matemática está presente nas mais diferentes atividades cotidianas, apesar disso, há diferenças  significativas  entre  a  Matemática  usada  no  dia  a  dia  e  a Matemática utilizada  em ambiente   escolar \cite{barbosa2020utilizacao}. Este distanciamento muitas vezes ocorre segundo \cite{goes2023ensino} para muitos alunos, aprender matemática se resume a memorizar algoritmos e fórmulas, seguindo as instruções do professor. Eles frequentemente enxergam a matemática como um conjunto de conceitos fixos, inquestionáveis e difíceis de entender, concebidos por indivíduos extremamente inteligentes. 

Por outro lado, muitos professores acreditam que os alunos realmente aprendem quando conseguem reproduzir o maior número de exercícios. No entanto, essa perspectiva é equivocada, pois na mera repetição, os alunos não interpretam o mundo ao seu redor e suas experiências. \cite{goes2023ensino}.

A dificuldade no ensino da matemática não é um problema novo. Diversos estudos, como o de  \cite{boaler2015s}, apontam que métodos tradicionais de ensino, baseados em memorização e repetição, muitas vezes não conseguem envolver os alunos de forma significativa, causando desmotivação. Um dos aspectos que têm contribuído para a desmotivação dos alunos em sala são as aulas  mecanizadas,  que  acontecem  de  forma  centrada  no  professor,  comquestões  repetitivas embasadas no ato de decorar fórmulas pré-estabelecidas e resolução de exercícios. Entretanto, vale  ressaltar  que  esse  modelo  mecanizado,  nos  dias  de  hoje,  ainda  é  uma  das  referências educacionais mais predominantes nas salas de aulasdo Brasil \cite{da2017contributo}.

Com o advento das tecnologias digitais, novas possibilidades surgiram para revolucionar o ambiente educacional. Ferramentas tecnológicas têm o potencial de tornar o ensino mais interativo e adaptado às necessidades individuais dos alunos. Segundo \cite{barbosa2020utilizacao} Na sociedade contemporânea, em meio às constantes mudanças e inovações no ambiente escolar, é essencial entender melhor como essas transformações afetam toda a comunidade escolar. Vale destacar que crianças e adolescentes estão profundamente imersos em um ambiente virtual, cercados por smartphones, computadores e outros dispositivos.

Assim, o avanço tecnológico veio de forma massiva, tomando espaço dentro da sala de aula como um recurso disponível para o ensino. Nessa perspectiva, ao se utilizar a tecnologia no ambiente escolar o professor se propõe a explorar as vantagens que este recurso pode trazer para a sala de aula, pois ao utilizá-la a seu favor,esta pode servir como uma forma de estimular o aluno ao aprendizado \cite{barbosa2020utilizacao}.

Dentro desse contexto, a gamificação aparece como tecnica inovadora e promissora, envolvendo elementos de design de jogos no contexto real, com o objetivo de engajar e motivar a aprendizagem. \cite{barbosa2021aplicativos} diz que a gamificação não é o ato de jogar ou criar um jogo; vai além ao implementar estratégias, estruturas e dinâmicas dos jogos em contextos não relacionados a jogos, integrando regras, metas, desafios, rankings e recompensas.

De acordo com \cite{santana2024intencionalidade} a competição amigável, bem como a cooperação, pode impulsionar o engajamento, pois os estudantes buscam melhorar seu desempenho e conquistar recompensas no ambiente de aprendizado. Nesta perspectiva esse trabalho tem objetivo utilizar de tecnicas gamificadas para auxiliar no ensino da matematica, demonstando assim a viabilidade desta metodologia de ensino. 

\section{Aplicabilidade e Motivação}\label{chp:aplic}

A dificuldade encontrada na docência do Ensino Fundamental I nas atribuições das aulas de matemática, tratam-se de abordagens que proporcionam alternâncias e intercorrências consideráveis. Os problemas, contratempos e adversidades apresentam-se desde os planejamentos pedagógicos por coordenadores em planos de cursos e sobrevêm chegando até os docentes em seus planos de aulas que não conseguem atingir com eficiência o entendimento e absorção da aprendizagem pelos discentes. 

Tanto o magistério quanto o alunato enfrentam objeções ao lidar com essa disciplina, procurando superar muitos obstáculos, ponderando que os incentivos deparam em grande parte dos estudantes que não consideram a matemática como algo notável ou concebível em suas vivências. Além disso, os estímulos para aprendizagem, aplicação das suas concepções e ideias são diretamente impactados. Igualmente a subjetividade e complexidão da matemática abrange concepções subjetivas e constantemente abstrusa, existindo a possibilidade do alunado encontrar obstruções para perceber e assimilar os conceitos, o que pode trazer o desapontamento. 

Da mesma forma a fixação e cognição dos estudantes focalizam em memorar os métodos e suas fórmulas em vez de tentar compreender os princípios implícitos subentendidos, impossibilitando a sua criatividade para solucionar determinadas questões com eficacia. Assim também as ferramentas necessárias para os educadores deparam-se com impedimentos para obter materiais instrutivos, compatíveis e metodologias de transmissão de conhecimentos eficientes, considerando a probabilidade de impactos negativos na natureza da orientação. Por analogia a aflição pela disciplina é perceptível em grande parte dos alunos que se apresentam ansiosos, pois o processo de aprender matemática oferece tendências que afetam a conduta e autoestima. 

Atualmente a maior parte dos jovens e até mesmo alguns adultos vivenciam a sensação de experienciar jogos com parâmetros físicos ou digitais por intermédio de videogames, computadores, smartphones, dominós, baralhos, dados etc. Subsiste neles alguma coisa indeterminada que atrai e magnetiza, tomando por completo a atenção do usuário. Que idealiza a sobrelevação, individualmente ou em equipe, incentivado com o resultado alcançado em suas fases, etapas, níveis e etc. Pois mesmo parecendo uma simples diversão, nas variadas formas e possibilidades de jogos se aplica energia espontaneamente. Além disso as atividades de entretenimentos, define a estruturação intrínseca mediante a diretrizes que tem por objetivo conduzir a atividade com procedimentos organizados.

\section{Contextualização}

A gamificação, que utiliza elementos de jogos para engajar e motivar os alunos, tem se destacado como uma abordagem eficaz para o ensino. A gamificação oferece oportunidades de aprendizado interativo, onde desafios, recompensas e feedbacks imediatos podem tornar a experiência educacional mais estimulante e atraente para as crianças.

A evolução das tecnologias tem contribuido significativamente na educação, principalmente em relação inclusão de pessoas, como pessoas com deficiência, proporcionando maior acessibilidade e autonomia.  Novos recursos e APIs que facilitam a interconexão e integração entre plataformas, têm possibilitado o desenvolvimento de aplicações e soluções tecnológicas que atendam a um número cada vez maior de usuários com necessidades especiais. Um exemplo disso é o aplicativo "Hand Talk", que disponibiliza um plugin para integrações de acessibilidade a deficientes auditivos, permitindo tradução automática para Língua Brasileira de Sinais.

Outras iniciativas importantes são ferramentas de leitura de tela, ampliação de conteúdo, contraste elevado e interfaces adaptáveis. Essas inovações têm o poder de reduzir barreiras, promover inclusão digital e transformar positivamente a vida de pessoas com deficiência. Como destaca  \cite{cordeiro2023inclusao} "Ao desempenhar um papel significativo na promoção da inclusão,permite a adaptação e personalização das ferramentas de acordo com as necessidades individuais de cada sujeito,com ou sem deficiência",.


\section{Objetivos}

Neste estudo acadêmico, a gamificação será abordada de forma abrangente, com reflexões e idealizações sobre um tema atual de grande importância na educação. O elemento conceitual investigativo deste objeto de pesquisa é altamente significativo, pois a gamificação requer um entendimento satisfatório para ser aplicada com eficiência, sensatez e seriedade. A temática será explorada em tópicos relevantes, investigando as concepções e ideias empregadas, os recursos e componentes adotados, e suas finalidades específicas. Assim, busca-se obter uma compreensão clara do conteúdo discutido neste trabalho de pesquisa e investigação científica, destacando a relevância e a importância considerável desta abordagem no campo acadêmico.

\subsection{Objetivo Geral}

 O objetivo geral deste trabalho é investigar como a gamificação pode ser implementada no ensino de matemática básica para tornar o aprendizado mais envolvente e eficaz. Para isso, foi desenvolvido o aplicativo Math.Pow, que utiliza elementos lúdicos, desafios interativos e recompensas para motivar e engajar alunos do 4º ano do Ensino Fundamental I. O estudo explora a viabilidade desta abordagem gamificada, apresentando fundamentos teóricos, etapas de desenvolvimento e resultados de validação com professores e alunos, com a esperança de superar barreiras no ensino da matemática e promover uma aprendizagem mais significativa.
 
\subsection{Objetivos Específicos}

\begin{itemize}
\item{Autenticar se o uso de jogos educacionais para o ensino de matemática é eficaz.
}
\item {Avaliar quais conteúdos são mais estruturantes para o ensino da matemática no fundamental I.
} 
\item {Desenvolver sequências didáticas com a metodologia da gamificação.
 }  
\item {Desenvolver um aplicativo com uso das técnicas de gamificação.
}
\item {Validar o uso do aplicativo com professores e estudantes.
}
\item {Auxiliar no ensino de matemática para alunos do ensino fundamental I.
}
\end{itemize}

\section{Estrutura do Trabalho}

Este trabalho está organizado em sete capítulos. No Capítulo 1 é apresentada a introdução, contextualização, objetivos gerais e específicos do projeto. O Capítulo 2 aborda a fundamentação teórica, com foco em gamificação e seu papel na educação matemática. O Capítulo 3 detalha a metodologia utilizada no desenvolvimento do trabalho. O Capítulo 4 apresenta o processo de construção da aplicação Math.Pow, incluindo requisitos, tecnologias e arquitetura. O Capítulo 5 descreve em detalhes o aplicativo Math.Pow e suas funcionalidades para o ensino de matemática. O Capítulo 6 apresenta os experimentos realizados, incluindo análise curricular, formulários avaliativos e resultados obtidos com professores e pais. Por fim, o Capítulo 7 traz as considerações finais do trabalho.


